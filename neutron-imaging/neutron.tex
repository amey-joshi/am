\documentclass{article}
\usepackage{amsmath, amsfonts, amssymb, amsthm}
\usepackage{url, graphicx}

\numberwithin{equation}{section}
\linespread{1.3}
\title{A Review of Neutron Imaging}
\author{To be filled}
\date{27-Sep-2020}
\begin{document}
\maketitle
\section{Introduction}\label{s1}
Nobody suspected the presence of a neutral particle in an atomic nucleus until
the 1930s. When Heisenberg developed matrix mechanics in 1925 and 
Schr\"{o}dinger wrote his celebrated equation, the nucleus was thought to be
made up of positively charged protons and a few negatively charged `internal
electrons'. The latter were supposed to be in present in sufficient numbers to
account for the mass and charge of the nucleus. However, this idea became
increasingly untenable with the development of quantum statistical mechanics
and an understanding of Fermions and Bosons. When Oscar Klein 
\cite{klein1929reflexion} applied the newly discovered Dirac equation to the
`internal electrons' in the nucleus he found that they should find it 
extremely easy to escape the potential barrier of the proton's electric field.
These difficulties were finally resolved when Chadwick reported 
\cite{chadwick1932existence}, the existence of an electrically neutral particle, called the neutron, in the atomic nucleus. 

\section{Physical properties of a neutron}\label{s2}
The mass of a free neutron is $1.674927471 \times 10^{-27}$ kg
\cite{taylor2007nist}. At non-relativistic speed $v$, its de Broglie wavelength
is \cite{anderson2008neutron}
\begin{equation}\label{s2e1}
\lambda = \frac{395.6}{v} \text{nm}
\end{equation}
making it an ideal source to explore condensed matter at the atomic scale.

Its mean square radius is $0.8 \times 10^{-15}$ m
\cite{bogdan2015particles}. It has no electric charge. However, it has
an intrinsic spin of $1/2$ (in units of $\hslash$)
\cite{basdevant2005fundamentals}. As a result, although it is unaffected by 
an electric field, it does respond to a magnetic field. In fact, a neutron 
responds to all four fundamental forces in Nature. Outside of
the nucleus, the neutron is unstable. It decays into a proton, an electron and
an electron neutrino with a half life of about $10$ minutes and $10$ seconds
\cite{Nakamura_2010}. The energy of decay is $0.782$ MeV \cite{heyde2004basic}.
A small fraction, about $0.001$, of free neutrons also produce gamma radiation 
while decaying. An even smaller fraction, about $4$ per million, of neutrons
produce electrons with an energy lesser than $13.6$ eV. As a result, they
stay stay bound to the proton forming a hydrogen atom. The fact that neutrons
decay cannot be ignored while using them for imaging. This feature of the
neutrons distingushes them from the other probes used for imaging like optical
photons, X-rays, gamma rays, infra-red radiation, sound waves and electrons. 
The latter probes are not known to decay.

\section{Sources of neutrons}\label{s3}
A neutron imaging device must have a source of neutrons the way an X-ray
imaging system has an X-ray tube to generate the X-rays. In this section we
will review a few sources of neutrons. Nuclear fission reactors produce a large
number of neutrons. A typical reactor has a fluence rate of the order of 
$10^{15}$ neutrons per $\text{cm}^{-2}$ per second. However, their energies are 
spread over a large range and the neutrons can be used only within the premises
of the reactor. Instead of relying on nuclear fission reactors, one can look
at nuclei that spontaneously decay to produce neutrons. A popular isotope of 
half-life of $2.6$ years is ${}^{252}$Cf. A typical source of this kind 
produces on an average of $10^{8}$ neutrons per second. Other examples of such
nuclei are ${}^{254}$Cf, ${}^{244}$Cm, ${}^{242}$Cm, ${}^{238}$Pu and 
${}^{232}$U.

One can go back to
Chadwick's experiment \cite{chadwick1932existence} to devise yet another source
of neutrons. Chadwick bombarded a slab of Beryllium with alpha radiation 
emitted from decaying Polonium nuclei. The nuclear reaction
\begin{equation}\label{s3e1}
{}_2^4\text{He} + {}_4^9\text{Be} \rightarrow {}_6^{12}\text{C}+{}_0^1\text{n}
\end{equation}
produced the first neutron detected in a laboratory. Some other reactions that 
also yeild neutrons are\cite{turner2008atoms}
\begin{eqnarray}
{}^2_1\text{H} + {}^3_1\text{H}&\rightarrow&{}_2^{4}\text{He}+{}_0^1\text{n}
\label{s3e2} \\
{}^2_1\text{H} + {}^2_1\text{H}&\rightarrow&{}_2^{3}\text{He}+{}_0^1\text{n}
\label{s3e3} \\
{}^2_1\text{H} + {}_6^{12}\text{C}&\rightarrow&{}_7^{13}\text{N}+{}_0^1\text{n}
\label{s3e4} \\
{}^1_1\text{p} + {}_3^1\text{H}&\rightarrow&{}_2^{4}\text{He}+{}_0^1\text{n}
\label{s3e5} \\
{}_1^1\text{p} + {}_3^7\text{Li}&\rightarrow&{}_4^{7}\text{Be}+{}_0^1\text{n}
 \label{s3e6} 
\end{eqnarray}
Each of these processes produce neutrons with a continuous energy spectrum. If
we need neutrons with narrow energy spread then we have to consider nuclear
reactions triggered by photons. Chadwick and Goldhaber 
\cite{chadwick1934nuclear} bombarded Deuterium nuclei with gamma rays of energy
exceeding $2.22$ MeV to trigger the reaction
\begin{equation}\label{s3e7}
{}^2_1\text{H} + \gamma \rightarrow {}^1_1 \text{H} + {}^1_0 n.
\end{equation}
Similarly, a photon of energy more than $1.67$ MeV can disintegrate Beryllium
into two Helium nuclei and a neutron.
\begin{equation}\label{s3e8}
{}^9_4\text{Be} + \gamma \rightarrow 2 {}^4_2\text{He} + {}^1_0 n.
\end{equation}
The $\gamma$-rays themselves come from radioactive ${}^{124}_{51}$Sb which 
spontaneously disintegrates as
\begin{equation}\label{s3e9}
{}^{124}_{51}\text{Sb} \rightarrow {}^{124}_{52}\text{Te} + e^{-1} + \gamma.
\end{equation}
Other commonly used isotopes to generate $\gamma$ radiation are ${}^{24}_{11}$
Na, ${}^{116}_{49}$In, ${}^{140}_{57}$La and ${}^{226}_{88}$Ra. A recent method
of generating neutrons involes spallation of heavy nuclei due to impact of 
very high energy protons. The term spallation means ejection of fragments from
a body due to extraordinarily large stress. In the case of nuclear spallation,
$\text{H}^{-}$ ions are accelerated to energy up to $1$ GeV before passing
them through a foil. The latter process strips the $\text{H}^{-}$ ion of its
two electrons and leaves behind a bare proton. The protons so generated are
made to pass at relativistic speeds into an accumulator ring. They are 
ejected out of the ring at a $60$ Hz into a tank of liquid Hg where they
trigger spallation of Hg nuclei. The greatest benefit of spallation is that
the resulting neutrons do not trigger a chain reaction as they do in the case
of nuclear fission reactors.

Neutrons of energies up to $0.1$ MeV are called `slow' neutrons, `Fast'
neutrons have energies in the range $10$ to $20$ MeV while those with even
higher energies are called relativistic neutrons.

\section{Interaction of neutrons with matter}\label{s4}
Neutrons being electrically neutral can travel significant distances in
condensed matter. The probability $\mu$ that a neutron interacts with matter
in unit distance is called its linear attenuation coefficient. It depends on
the energy of the neutron and the properties of the condensed matter through
which it is traveling. If $N(x)$ is the number of neutron that have traversed
a distance $x$ in the condensed matter without any interaction then the number
of neutrons $dN$ that will interact in the distance interval $x$ and $x + dx$
is
\begin{equation}\label{s4e1}
dN = -\mu Ndx.
\end{equation}
The solution of this equation is 
\begin{equation}\label{s4e2}
N = N_0 e^{-\mu x},
\end{equation}
where $N_0 = N(x = 0)$ is the number of neutrons incident on a beam of the
material. Neutrons interact only with atomic nuclei. In an elastic collision
they transfer a portion of their kinetic energy to the nuclei. A series of
elastic collisions leads to slowing of neutrons, also called their moderation.
Eventually, the kinetic energy of the neutron drops down to the level of the
thermal energy of the surroundings and it gets absorbed by a nucleus. Neutrons
can also undergo inelastic collisions with nuclei. In this case, the neutron
transfers a portion of its energy to the incident nucleus, raising it to an 
excited state. Even in this case, neutron meets the same fate of losing energy
up to the level of thermal energy and getting absorbed. Materials rich in 
protons, that is H${}^+$ ions, show an overwhelmingly large proportion of
elastic collisions because the bare proton does not have excited states. The
de-energized neutron is absorbed leading the formation of a Deuterium atom
\begin{equation}\label{s4e3}
{}_1^1H + {}^1_0 n \rightarrow {}^2_1 H + \gamma.
\end{equation}
The energy of the resulting $\gamma$-radiation is $2.22$ MeV. The precise 
energy of this radiation is used to detect the presence of neutrons.
In
materials with other nuclei, a plot of the neutron collision cross sections 
with neutron energies show pronounced peaks and troughs at the nuclear 
transition levels. A notable aspect of elastic collisions is that a neutron of
mass $m_n$ can transfer its kinetic energy to a nucleus of mass $m$ to the
extent of
\begin{equation}\label{s4e4}
Q = \frac{4m_nmE_n}{(m + m_n)^2},
\end{equation}
where $E_n$ is the energy of the neutrons. In a material rich with protons,
$m = m_n$ and $Q = 1$. Light nuclei are thus far more effective in slowing
neutrons than heavier ones. Elastic collisions of neutrons with heavy nuclei
change the direction of motion of nuclei while absorbing only a small portion
of their kinetic energy.

The interaction of neutrons with matter are used to develop neutron imaging
techniques. The elastic scattering gives information about the atomic 
structure. Neutron diffraction techniques have been used to study solids. On
the other hand, inelastic scattering reveals information about collective
excitations. Interactions of both kinds give rise to contrast that is used
for imaging.

\section{Detection of neutrons}\label{s5}
Neutrons are chargeless and their magnetic moment is too low to be used for
reliably detecting their presence. They also have a small electric dipole 
moment but it has not been measured in a laboratory. Free neutrons decay to
produce a proton and an electron but its half life of around $10$ minutes is
too long to permit the detection of neutrons through its decay products. That
leaves us with just one option, consider nuclear reactions triggered by the
neutrons and measure their results to estimate the number and energy of the
neutrons. Since the imaging applications involve mostly slow neutrons, we will
focus our attention on their detection. 

Slow neutrons are detected by the nuclear reactions they trigger in ${}^{10}_5$
B, ${}^6_3$Li and ${}^3_2$He. The first two of these produce alpha particles
while the last one produces hydrogen atom. One can use non-relativistic energy
and momentum conservation laws to show that the energy of the alpha particle
in these reactions is \cite{turner2008atoms}
\begin{equation}\label{s5e1}
E = \frac{MQ}{m + M},
\end{equation}
where $m$ is the mass of the alpha particle, $M$ that of the daughter 
nucleus and $Q$ is the energy released in the reaction. The resulting alpha
particles are detected using proportional counters. Proportional counters
measure only the number of neutrons, not their energy spectrum. The reactions
used for detection are \cite{crane1991neutron}
\begin{equation}\label{s5e2}
{}^{10}_5\text{B}+{}^1_0\text{n} \rightarrow \begin{cases}
{}^7_3\text{Li}^\star + {}^4_2\text{He} + 2.31\text{ MeV} \\
{}^7_3\text{Li} + {}^4_2\text{He} + 2.79\text{ MeV}, 
\end{cases}
\end{equation}
where the $\star$ superscript in the first reaction indicates that the nucleus
is in an excited state. The other reactions are
\begin{eqnarray}
{}^6_3\text{Li}+{}^1_0\text{n}&\rightarrow&{}^3_1\text{H} + {}^4_2\text{He} +
4.78\text{ MeV} \label{s5e3} \\
{}^3_2\text{He}+{}^1_0\text{n}&\rightarrow&{}^3_1\text{H} + {}^1_1\text{H} +
0.765\text{ MeV} \label{s5e4} 
\end{eqnarray}
We mentioned previously that the collision between a proton and a neutron is
always elastic. Further, since their masses are almost equal, it is possible
for the neutron to transfer all its kinetic energy to the proton. The proton
is then detected by scintillators. Organic compounds are usually rich in 
protons and detectors based on this principle use them in crystalline, 
amorphous or even liquid forms. 

Bubble detectors are inexpensive and portable neutron detecting devices. They
consist of a transparent polymer with a large number of tiny droplets of 
a superheated liquid. The passage of a charged particle through such droplets
results in a rapid phase change from liquid to vapor. The transparent droplets
become visible bubbles which are counted either manually or electronically.
\section{Neutron dosimetry and shielding}\label{s6}
Radiation Dosimetry is the measurement of the `dose' of radiation absorbed by
a biological sample. Dosimetry is important for the safety of people working
with radiation. Biological materials are rich in protons. When they are
exposed to neutrons, the elastic collision of the neutrons and the protons
leads to a transfer of a substantial amount of energy to the protons resulting
in breaking of chemical bonds. Neutrons damage biological samples more than
a comparable dose of electrons or $\gamma$ rays. The amount of energy absorbed
by $1$ kg of a material is called the `absorbed dose'. It is measured in grays,
where $1$ gray is equal to $1$ Jkg${}^{-1}$. It is measured and specified at 
every point of the exposed sample. Before getting into the details of neutron
dosimetry, we introduce a few physical concepts.

The restricted stopping power of a target is the the linear rate of energy of
energy lost,  $(-dE/dx)_\Delta$, due to collisions alone and in which the
energy transfer does not exceed $\Delta$. If $\mu$ is the probability per unit
distance of travel that a neutron collides with a proton and if $W(Q)dQ$ is the 
probability that a given collision will result in an energy loss between $Q$
and $Q + dQ$ then
\begin{equation}\label{s6e1}
-\left(\frac{dE}{dx}\right)_\Delta = \mu\int_{Q_{\text{min}}}^\Delta QW(Q)dQ,
\end{equation}
where $Q_{\text{min}}$ is the minimum energy needed for excitation of the 
target \cite{turner2008atoms}. The stopping power depends on both the 
incident radiation and the absorbing sample. Radiation with high restricted
stopping power is more dangerous than the one with low restricted stopping 
power. The product of the absorbed dose $D$ and a quality factor $Q$ (not 
related to the $Q$ in equation \eqref{s6e1} is called the `dose equivalent'
$H$ of the sample. The absorbed dose is also a point function of the sample.
That is, it can be specified at every point of the sample. The SI unit of 
absorbed dose is sievert.

Neutrons are best shielded by material rich in protons like Iron, water, 
paraffin, concrete and graphite, in decreasing order of attenuation length
\cite{turner2008atoms}.

\section{Optics of neutron beams}\label{s7}
Typical neutron sources have size of the order of $10$ cm in all directions.
They are enclosed in moderators to produce a quasi-Maxwell distribution
\cite{anderson2008neutron}. Their brightness is defined as the flux of neutrons
per solid angle per unit wavelength. The flux itself is the number of neutrons
crossing a unit area. One can treat the neutrons using the principles of
geometrical optics. The phase space of analytical mechanics can be carried over
to optics and is called etendue. Consider an infinitesimal surface element 
$dS$ with normal $\hat{n}$ pointing in a medium of refractive index $n$. If
the surface $dS$ is crossed by neutrons confined to a solid angle $d\Omega$
then the etendue of neutron beam is
\begin{equation}\label{s7e1}
dG = n^2\hat{n}\cdot\hat{n}_b dSd\Omega,
\end{equation}
where $\hat{n_b}$ is the normal along the axis of the cone of $d\Omega$ 
\cite{markvart2007thermodynamics}. The way the volume of the phase space is
conserved, one can also conclude that the volume of an etendue is conserved. 
The optical analogue of Liouville's theorem places a limit on the efficiency
of a neutron source \cite{anderson2008neutron}, 
\cite{markvart2007thermodynamics}. If one want to focus a neutron beam from
a source of size $w_1$ on a spot of size $w_2 < w_1$ then one must allow the
beam to diverge in directions normal to the line joining the source and the
spot.

Neutron beams can also be transported away from the source using wave guides.
They are channels coated with Nickel which has a critical angle of $0.10^\circ$
per $\AA$ neutron wavelength. Where higher critical angles than Nickel are 
needed super-mirrors are used. Critical angles for super-mirrors can be $4$
to $6$ times more than Nickel. Oftentimes it is beneficial to carry neutrons
away from the source to prevent their contamination with fast neutrons which
are also emitted from the source. Allowing the neutrons a longer flight time
through wave guides also improves the resolution of pulsed-source instruments.
The latter quantity is proportional to $\Delta t/t$, where $t$ is the total
time of flight and $\Delta t$ is the time width of the pulse in the moderator.

A neutron source produces neutrons with a wide range of wavelengths. A device
that allows us to select one among them is called a monochromator. Crystal
monochromators are based on Bragg's law
\begin{equation}\label{s7e2}
n\lambda = 2d\sin\theta,
\end{equation}
where $n$ is the `order' of diffraction, $\lambda$ is the wavelength of 
neutrons, $d$ is the distance between the lattice planes and $\theta$ is the
angle between the crystal and the incident beam. (The angle of incidence is
$\pi/2 - \theta$.) Once a crystal is chosen, $d$ is fixed. One can select
the wavelength of choice by changing $\theta$. The intensity of the diffracted
beam reduces for higher order of diffractions. Therefore, monochromators us
the first order beam.

The direction of propagation of neutron beams can be changed by refraction.
The refractive index of neutron beams is given by the Fermi thin-slab formula
\cite{warner1985neutron}
\begin{equation}\label{s7e3}
n = 1 - \frac{2\pi\rho b}{k_0^2},
\end{equation}
where $b$ is the neutron scattering length, $\rho$ is the number density of
the scattering nuclei and $k_0$ is the propagation constant in free space.
A refractive index lesser than $1$ indicates that one must use a concave lens
to focus a neutron beam. Although a neutron is electrically neutral, it does
have spin. Therefore, it can be deflected using a magnetic field. If 
$\vec{r}(t)$ is the trajectory of a neutron beam then its deflection in a
magnetic field $\vec{B}$ is given by
\begin{equation}\label{s7e4}
\frac{d^2\vec{r}}{dt^2} = \pm\frac{\mu}{m}\nabla|\vec{B}|,
\end{equation}
where $\mu$ is the neutron's magnetic moment and $m$ is its mass.

The existence of spin states of neutrons allows us to describe beams of 
neutrons by their polarization states. Being a spin-half particle, a neutron
can exist in one of $|+\rangle$ of $|-\rangle$ states. If a beam has $N_+$
neutrons in $|+\rangle$ and $N_-$ neutrons in $|-\rangle$ states then its
polarization is defined to be
\begin{equation}\label{s7e5}
P = \frac{N_+ - N-}{N_+ + N_-}.
\end{equation}
It is equally common to express beam polarization in terms of flipping ratio
as
\begin{equation}\label{s7e6}
F = \frac{N_+}{N_-} = \frac{1 - P}{1 + P}.
\end{equation}
When the nuclear and the magnetic structure factors of a crystal are equal in
magnitude, the total structure factor vanishes for one of the two spin states.
As a result, the Bragg reflection from the crystal plance produces a beam with
almost all neutrons in the remaining polarization states. The most commonly
used crystal to achieve neutron polarization is the [111] reflection of 
Heusler alloy \cite{anderson2008neutron}. This technique is most suitable for
steady state source of neutrons \cite{williams1988polarized}. In the case of
a pulsed beam of neutrons it is common to use resonance absorption filters.
When an unpolarized neutron beam whose energy corresponds to a nuclear
resonance is shone on a target of polarized nuclear spins, one of the 
neutron spins is absorbed while the other one is transmitted 
\cite{williams1988polarized}.

In order to use a neutron beam for imaging the neutrons emerging from a source
must be collimated into a stream of particles traveling mostly along a single
direction. Being electrically neutral, they cannot be focused using an electric
field. Their small magnetic moment requires a very large field gradient to 
change their direction of propagation. A simpler way of achieving collimation
is to collect neutrons in a tube lined with a material which is opaque to 
neutrons. Collimators have a small entrance aperture and a larger exit allowing
a large neutron flux.

\section{2-d neutron imaging}\label{s8}
2-d neutron imaging is similar to the more commonly known 2-d X-ray imaging.
When X-rays are incident on a target and when then the intensity of the 
transmitted beam is analyzed one observes that different portions of the
latter have different intensities. This is because different portions of the
transmitted beam have emerged from different parts of the target each one
having their own attenuation properties. 2-d neutron imaging works in exactly
the same manner. Attenuation of X-rays depends on the local electronic
properties of the target while that of neutron depends on the local nuclear
properties. That is why, the two imaging techniques are often considered to
provide complementary information about the target. X-rays provide information
about the bonding or structural properties while neutrons indicate the 
elemental composition. The set up for 2-d neutron imaging is quite simple. It 
has a source of neutrons surrounded by a moderator to thermalize them, a 
collimator to coax the particles along a single direction and a detector to
visualize the image.

The intensity of the incident and the transmitted beams is given by equation
\eqref{s4e2}. If $I_0$ and $I$ are the intensities of the incident and the
transmitted beams then the degree of darkness of the film used to visualize
the image is
\begin{equation}\label{s8e1}
D_e = \ln\frac{I_0}{I}.
\end{equation}
The film brightens most where the transmitted intensity is the highest. In the
case of a digital receptor, the grey level value $G$ is
\begin{equation}\label{s8e2}
G = Cn + G_0,
\end{equation}
where $C$ is a constant, $n$ is the neutron density and $G_0$ is another 
constant that indicates the gray level due to noise in the circuits. The 
presence of $G_0$ suggests that $G \ne 0$ even in the absence of neutron flux.
This method suffers from the disadvantage of catching signals not only from
the transmitted neutrons but from other ambient radiation as well. As a result,
the images often are blurred. Nevertheless, it is used in situation where the
sharpness of the image is not critical. For example, in the non-destructive
testing in materials, it sufficient to know if the material has contaminants,
defects, lubrication etc. It is also used in the investigation of two-phase
flows, flows through porous media \cite{anderson2008neutron}, detection of
hydrgenous materials and moisture \cite{richards2004neutron}.

\section{Neutron tomography}\label{s9}
Tomography is an imaging technique in which the image of a target is constructed
by combining the images of its slices. The most usage is in medical imaging
in the form of CT (Computerized Tomography) scans. X-rays are used over a 
large number of slices of the human body. The `image' produced by each slice
is combined with others to get complete information of the portion of interest.
The way 2-d neutron imaging is an adaptation of 2-d X-ray imaging, neutron
tomography is applying the principles of X-ray tomography to neutron imaging. 
Consider a target $T$ whose image we are are interested in. We place the origin
of the cartesian coordinate system at a suitable point in $T$. We consider $T$
to be composed of a large number of slices whose normal is along the $z$-axis.
We know from equation \eqref{s4e2} that the attenuation of intensity of an
incident neutron beams by a slice $S$ is described by the equation
\begin{equation}\label{s9e1}
I = I_0\exp\left(-\int_S\mu(x, y)ds\right).
\end{equation}
Since the target is not a homogenous object, we cannot consider it to have a
uniform attenuation $\mu$ as we did in equation \eqref{s4e2}. The parameter
$s$ is the coordinate along the direction of propagation of neutrons in the
target. We can rearrange equation \eqref{s9e1} and define a function
\begin{equation}\label{s9e2}
p(t, \theta) = \ln\left(\frac{I}{I_0}\right) = -\int_S\mu(x, y)ds.
\end{equation}
The coordinate $t$ runs perpendicular to the coordinate $s$. One can write 
this in terms of an integral that extends over all $s$ as
\begin{equation}\label{s9e3}
p(t, \theta) = 
-\int_{-\infty}^\infty \mu(x, y)\delta(t - x\cos\theta - y\sin\theta)ds.
\end{equation}
The function $p(t, \theta)$ is called the Radon transform of the slice. It is
taken along the line $t = x\cos\theta + y\sin\theta$, expressed in the 
normal form \cite{gonzales1992digital}. It is the projection of the function
$f$ on the line $t = x\cos\theta + y\sin\theta$. It can be shown using the
Fourier Slice Theorem \cite{bracewell1956strip} that a Fourier transform of
$p(t, \theta)$ is the same as the slice of the 2-d Fourier transform of the
target. Thus, taking the 2-d inverse Fourier transform of the result of the
first operation will give us the original function. The number of such
projections is determined by Shannon's sampling theorem. 

Neutron tomography has the same advantages over neutron radiography that CT
scans has over X-ray images. Neutron radiography gives information about the
total attenuation of a neutron beam along its path through the target. It 
cannot give information about the actual distribution of materials across that
path. One of the earliest use cases of neutron tomography was in finding 
concentration of hydrogen in metal castings. The presence of hydrogen is an
indicator of the loss of mechanical properties due to embrittlement. We know
that the collision cross-section of neutron with hydrogen is very large and
that neutron imaging is a powerful tool to investigate hydrogenous materials. 
Neutron tomography techniques were able to give accurate information 
about hydrogen concentration as low as $200$ ppm \cite{richards2004neutron}.
Tomographic techniques are also suitable for investigating objects that do not
have a substantial plane surface like the blade of a turbine or an aircraft 
engine \cite{richards2004neutron}.

\section{Phase-contrast imaging}\label{s10}
When electromagnetic radiation passes from one medium to another it undergoes
both refraction and attenuation. The phenomena can be described together 
through the complex refractive index \cite{heald2012classical} where the
real part describes the refraction and the imaginary part the attenuation. 
The propagation of neutron beams through media too can be expressed by a 
complex refractive index. The 2-d imaging and the tomographic techniques
considered in the previous sections considered images acquired due to 
attenuation. In this section we shall briefly considered imaging due to phase
contrast. Equation \eqref{s7e3} indicated that the refractive index of neutrons
is less than $1$. Therefore, it is customary to express the complex refractive
index as
\begin{equation}\label{s10e1}
n(\vec{x}) = 1 - \delta(\vec{x}) + i\beta(\vec{x}),
\end{equation}
where the dependence on $\vec{x}$, the position in the medium, allows for the
inhomogeneous nature of the real and imaginary parts of the refractive index.
Aluminum and Silicon have very small values $\beta$. Therefore, they hardly
attenuate a neutron beam and cannot produce a 2-d image or a tomogram of 
sufficient contrast. On the other hand, their $\delta$ values are $5$ orders
of magnitude larger and can yield excellent phase-contrast images 
\cite{anderson2008neutron}. Titanium and Copper have very similar $\beta$
values and therefore cannot be differentiated by their attenuation properties.
Their $\delta$ values are of opposite signs. Titanium with $\delta=-5.0 \times
10^{-6}$ makes neutron beams behave somewhat like electromagnetic radiation
by making the real part of the refractive index slightly greater than one
\cite{anderson2008neutron}.

The phase differences caused by the real part of the refractive index are
detected by interference \cite{zawisky2004neutron}. A monochromatic beam of
neutrons is coherently split into a reference beam and an object beam. The 
object beam is passed through the target and suffers a phase shift (and 
attenuation). After it emerges from the target it is superposed with the 
reference beam. The interference between the two is captured in a digital
neutron camera.

Phase differences can either be recorded as a 2-d image or a tomogram. In the
latter case, the 3-d distribution of the refractive index of the target is 
reconstructed. The phase differences can either be represented as the 
differences in their values or the gradient of phase shift or the laplacian of
the phase shift \cite{anderson2008neutron}.

\section{Larmor precession based imaging}\label{s11}
A magnetic dipole of moment $\vec{\mu}$ experiences a torque $\vec{\tau} = 
\vec{\mu} \times \vec{B}$ in a magnetic field $\vec{B}$. If $\gamma$ is the
gyromagnetic ratio of the dipole then $\vec{\mu} = \gamma\vec{L}$, $\vec{L}$
being its angular momentum. Since
\begin{equation}\label{s11e1}
\frac{d\vec{L}}{dt} = \vec{\tau} = \gamma\vec{L} \times \vec{B},
\end{equation}
when $\vec{B}$ is not parallel (or anti-parallel) to $\vec{L}$, the vector 
$\vec{L}$ precesses about $\vec{B}$ with a frequency $\omega = \gamma B$, 
called the Larmor frequency. When a polarized beam of neutrons is made to
pass through a magnetic material, the spin vector of the neutrons undergoes
Larmor precession in the internal magnetic field of material. The angle of 
spin changes by an amount that depends on the magnitude of the magnetic field
and the time the neutron was present in it. The change in the spin orientation
can be measured by a spin analyzer. Different regions of the material show
different changes in spin orientation. By mapping the change in spin to grey
scale one can create a contrast between these regions 
\cite{kardjilov2011neutron}. This is the principle behind Larmor precession
based neutron imaging. If the magnetic fields are sufficiently low so that
the Larmor tilt is never more than $\pm\pi$, quantitative 2-d and 3-d imaging
is possible. It was used to study the trapping of magnetic flux in type-I 
superconductors \cite{kardjilov2008three}. This idea can be extended to 
examine the how a ferromagnet becomes a paramagnet past the Curie temperature.
When a beam of polarized neutrons falls on a ferromagnet, the randomly oriented
magnetic domains of the ferromagnet depolarize the beam. When the same material
is heated, the domains begin to weaken and beyond the Curie temperature they
disappear altogether. The polarized beam is not transmitted undisturbed. This
technique can be used to study the change in magnetic properties of materials
in response to changing stress, pressure or applied electromagnetic fields
\cite{kardjilov2011neutron}.

\section{Spin-echo based imaging}\label{s12}
Consider a beam of polarized nucleons (neutrons, protons, or nuclei with an odd
number of nucleons) passes through a region of homogenous magnetic field its
spin vector tilts with respect to the original direction because of Larmor
precession. When the nucleons emerge from the region, they all emerge with the
same tilt. However, as time goes by, different nucleons have a different angle
of tilt. This happens because of natural spin relaxation and due to 
inhomogeneities in the surroundings of the nucleons. If the nucleons are 
subjected to a field that flips the spin by $\pi$ then after a while all spins
that had dispersed due to inhomogeneities once again orient themselves together
creating an `echo'. The magnitude of the echo is not the same as that of the
original beam because of the natural spin relaxation. This is the principle
behind the 'spin-echo' phenomenon used in nuclear magnetic resonance 
spectroscopy and magnetic resonance imaging.

In the nuclear spin echo experiment this arrangement is enhanced with an 
additional region of uniform magnetic field, identical to the first one
\cite{mezei1972neutron}. When a polarized neutron enters the first region
of a uniform magnetic field, its spin vector tilts by an angle, say $\varphi$
radians. Different neutrons experience a different tilt depending on the 
time they spent in this region, which in turn depends on their velocity. Now
suppose that the neutron beam is made to pass through a target. Each neutron
will see a different local magnetic field in the target and will react to it
in its unique way. The beam of neutrons is further defocused because of these
inhomogeneities. It is then made to pass through a $\pi$ flipper. The 
defocusing that had occurred due to inhomogeneities gets a chance to correct
itself leading to a spin echo. A second region of a uniform magnetic field
brings all the spins to the direction parallel to the originally polarized
neutrons. The inelastic scattering of the neutrons in the target make some of
them lose more energy than others. The second region of homogeneous magnetic
field fails to bring such neutrons back to their original polarization state.
The difference in the direction of polarization creates the contrast for
neutron spin-echo imaging.

Neutron spin-echo imaging is used with small angle neutron scattering in the
investigation of soft matter. The latter technique provides information about
the structure. However, the mechanical properties depends on molecular 
mobilities and constraints of motion \cite{monkenbusch2007high}, where the
energy differences are of the order of $0.7 \mu$eV. Such small changes can be
resolved only through neutron spin-echo imaging.

\section{Imaging of biologically important structures}\label{s13}
Biological targets have an extremely rich mixture of molecules. A knowledge of
the presence of certain trace elements in the targets reveals important 
information about its functioning, health and pathology. Few imaging techniques
can accurately provide information about the elements. The chemical shift 
imaging in magnetic resonance imaging provides information at the molecular
level. It provides little information about the elements present. This is where
neutron imaging comes into the picture. When neutrons undergo an inelastic
scattering with the nuclei of the target, they nuclei get into an excited
state. They get back to their ground state after emitting a photon. We now 
have a detailed knowledge of the nuclear energy levels. By measuring the energy
of the emitted photon we can guess the nucleus, and thus the element, which
might have produced it. This is the principle behing the neutron stimulated
emission computed tomography (NSECT) imaging.
\section{Conclusion}\label{s14}
\bibliography{neutron}
\bibliographystyle{plain}
\end{document}
