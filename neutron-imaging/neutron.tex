\documentclass{article}
\usepackage{amsmath, amsfonts, amssymb, amsthm}
\usepackage{url, graphicx}

\numberwithin{equation}{section}
\linespread{1.3}
\title{A Review of Neutron Imaging}
\author{To be filled}
\date{27-Sep-2020}
\begin{document}
\maketitle
\section{Introduction}\label{s1}
Nobody suspected the presence of a neutral particle in an atomic nucleus until
the 1930s. When Heisenberg developed matrix mechanics in 1925 and 
Schr\"{o}dinger wrote his celebrated equation, the nucleus was thought to be
made up of positively charged protons and a few negatively charged `internal
electron'. The latter were supposed to be in present in sufficient numbers to
account for the mass and charge of the nucleus. However, this idea became
increasingly untenable with the development of quantum statistical mechanics
and an understanding of Fermions and Bosons. When Oscar Klein 
\cite{klein1929reflexion} applied the newly discovered Dirac equation to the
`internal electrons' in the nucleus he found that they should find it 
extremely easy to escape the potential barrier of the proton's electric field.
These difficulties were finally resolved when Chadwick reported 
\cite{chadwick1932existence}, the existence of an electrically neutral particle, called the neutron, in the atomic nucleus. 

\section{Physical properties of a neutron}\label{s2}
The mass of a free neutron is $1.674927471 \times 10^{-27}$ kg
\cite{taylor2007nist}. Its mean square radius is $0.8 \times 10^{-15}$ m
\cite{bogdan2015particles}. It has no electric charge. However, it has
an intrinsic spin of $1/2$ (in units of $\hslash$)
\cite{basdevant2005fundamentals}. As a result, although it is unaffected by 
an electric field, it does respond to a magnetic field. In fact, a neutron 
responds to all four fundamental forces in Nature. Outside of
the nucleus, the neutron is unstable. It decays into a proton, an electron and
an electron neutrino with a half life of about $10$ minutes and $10$ seconds
\cite{Nakamura_2010}. The energy of decay is $0.782$ MeV \cite{heyde2004basic}.
A small fraction, about $0.001$, of free neutrons also produce gamma radiation 
while decaying. An even smaller fraction, about $4$ per million, of neutrons
produce electrons with an energy lesser than $13.6$ eV. As a result, they
stay stay bound to the proton forming a hydrogen atom. The fact that neutrons
decay cannot be ignored while using them for imaging. This feature of the
neutrons distingushes them from the other probes used for imaging like optical
photons, X-rays, gamma rays, infra-red radiation, sound waves and electrons. 
The latter probes are not known to decay.

\section{Sources of neutrons}\label{s3}
A neutron imaging device must have a source of neutrons the way an X-ray
imaging system has an X-ray tube to generate the X-rays. In this section we
will review a few sources of neutrons. Nuclear fission reactors produce a large
number of neutrons. A typical reactor has a fluence rate of the order of 
$10^{15}$ neutrons per $\text{cm}^{-2}$ per second. However, their energies are 
spread over a large range and the neutrons can be used only within the premises
of the reactor. Instead of relying on nuclear fission reactors, one can look
at nuclei that spontaneously decay to produce neutrons. A popular isotope of 
half-life of $2.6$ years is ${}^{252}$Cf. A typical source of this kind 
produces on an average of $10^{8}$ neutrons per second. Other examples of such
nuclei are ${}^{254}$Cf, ${}^{244}$Cm, ${}^{242}$Cm, ${}^{238}$Pu and 
${}^{232}$U.

One can go back to
Chadwick's experiment \cite{chadwick1932existence} to devise yet another source
of neutrons. Chadwick bombarded a slab of Beryllium with alpha radiation 
emitted from decaying Polonium nuclei. The nuclear reaction
\begin{equation}\label{s3e1}
{}_2^4\text{He} + {}_4^9\text{Be} \rightarrow {}_6^{12}\text{C}+{}_0^1\text{n}
\end{equation}
produced the first neutron detected in a laboratory. Some other reactions that 
also yeild neutrons are\cite{turner2008atoms}
\begin{eqnarray}
{}^2_1\text{H} + {}^3_1\text{H}&\rightarrow&{}_2^{4}\text{He}+{}_0^1\text{n}
\label{s3e2} \\
{}^2_1\text{H} + {}^2_1\text{H}&\rightarrow&{}_2^{3}\text{He}+{}_0^1\text{n}
\label{s3e3} \\
{}^2_1\text{H} + {}_6^{12}\text{C}&\rightarrow&{}_7^{13}\text{N}+{}_0^1\text{n}
\label{s3e4} \\
{}^1_1\text{p} + {}_3^1\text{H}&\rightarrow&{}_2^{4}\text{He}+{}_0^1\text{n}
\label{s3e5} \\
{}_1^1\text{p} + {}_3^7\text{Li}&\rightarrow&{}_4^{7}\text{Be}+{}_0^1\text{n}
 \label{s3e6} 
\end{eqnarray}
Each of these processes produce neutrons with a continuous energy spectrum. If
we need neutrons with narrow energy spread then we have to consider nuclear
reactions triggered by photons. Chadwick and Goldhaber 
\cite{chadwick1934nuclear} bombarded Deuterium nuclei with gamma rays of energy
exceeding $2.22$ MeV to trigger the reaction
\begin{equation}\label{s3e7}
{}^2_1\text{H} + \gamma \rightarrow {}^1_1 \text{H} + {}^1_0 n.
\end{equation}
Similarly, a photon of energy more than $1.67$ MeV can disintegrate Beryllium
into two Helium nuclei and a neutron.
\begin{equation}\label{s3e8}
{}^9_4\text{Be} + \gamma \rightarrow 2 {}^4_2\text{He} + {}^1_0 n.
\end{equation}
The $\gamma$-rays themselves come from radioactive ${}^{124}_{51}$Sb which 
spontaneously disintegrates as
\begin{equation}\label{s3e9}
{}^{124}_{51}\text{Sb} \rightarrow {}^{124}_{52}\text{Te} + e^{-1} + \gamma.
\end{equation}
Other commonly used isotopes to generate $\gamma$ radiation are ${}^{24}_{11}$
Na, ${}^{116}_{49}$In, ${}^{140}_{57}$La and ${}^{226}_{88}$Ra. A recent method
of generating neutrons involes spallation of heavy nuclei due to impact of 
very high energy protons. The term spallation means ejection of fragments from
a body due to extraordinarily large stress. In the case of nuclear spallation,
$\text{H}^{-}$ ions are accelerated to energy up to $1$ GeV before passing
them through a foil. The latter process strips the $\text{H}^{-}$ ion of its
two electrons and leaves behind a bare proton. The protons so generated are
made to pass at relativistic speeds into an accumulator ring. They are 
ejected out of the ring at a $60$ Hz into a tank of liquid Hg where they
trigger spallation of Hg nuclei. The greatest benefit of spallation is that
the resulting neutrons do not trigger a chain reaction as they do in the case
of nuclear fission reactors.

Neutrons of energies up to $0.1$ MeV are called `slow' neutrons, `Fast'
neutrons have energies in the range $10$ to $20$ MeV while those with even
higher energies are called relativistic neutrons.

\section{Interaction of neutrons with matter}\label{s4}
Neutrons being electrically neutral can travel significant distances in
condensed matter. The probability $\mu$ that a neutron interacts with matter
in unit distance is called its linear attenuation coefficient. It depends on
the energy of the neutron and the properties of the condensed matter through
which it is traveling. If $N(x)$ is the number of neutron that have traversed
a distance $x$ in the condensed matter without any interaction then the number
of neutrons $dN$ that will interact in the distance interval $x$ and $x + dx$
is
\begin{equation}\label{s4e1}
dN = -\mu Ndx.
\end{equation}
The solution of this equation is 
\begin{equation}\label{s4e2}
N = N_0 e^{-\mu x},
\end{equation}
where $N_0 = N(x = 0)$ is the number of neutrons incident on a beam of the
material. Neutrons interact only with atomic nuclei. In an elastic collision
they transfer a portion of their kinetic energy to the nuclei. A series of
elastic collisions leads to slowing of neutrons, also called their moderation.
Eventually, the kinetic energy of the neutron drops down to the level of the
thermal energy of the surroundings and it gets absorbed by a nucleus. Neutrons
can also undergo inelastic collisions with nuclei. In this case, the neutron
transfers a portion of its energy to the incident nucleus, raising it to an 
excited state. Even in this case, neutron meets the same fate of losing energy
up to the level of thermal energy and getting absorbed. Materials rich in 
protons, that is H${}^+$ ions, show an overwhelmingly large proportion of
elastic collisions because the bare proton does not have excited states. The
de-energized neutron is absorbed leading the formation of a Deuterium atom
\begin{equation}\label{s4e3}
{}_1^1H + {}^1_0 n \rightarrow {}^2_1 H + \gamma.
\end{equation}
The energy of the resulting $\gamma$-radiation is $2.22$ MeV. The precise 
energy of this radiation is used to detect the presence of neutrons.
In
materials with other nuclei, a plot of the neutron collision cross sections 
with neutron energies show pronounced peaks and troughs at the nuclear 
transition levels. A notable aspect of elastic collisions is that a neutron of
mass $m_n$ can transfer its kinetic energy to a nucleus of mass $m$ to the
extent of
\begin{equation}\label{s4e4}
Q = \frac{4m_nmE_n}{(m + m_n)^2},
\end{equation}
where $E_n$ is the energy of the neutrons. In a material rich with protons,
$m = m_n$ and $Q = 1$. Light nuclei are thus far more effective in slowing
neutrons than heavier ones. Elastic collisions of neutrons with heavy nuclei
change the direction of motion of nuclei while absorbing only a small portion
of their kinetic energy.
\section{Detection of neutrons}\label{s5}
Neutrons are chargeless and their magnetic moment is too low to be used for
reliably detecting their presence. They also have a small electric dipole 
moment but it has not been measured in a laboratory. Free neutrons decay to
produce a proton and an electron but its half life of around $10$ minutes is
too long to permit the detection of neutrons through its decay products. That
leaves us with just one option, consider nuclear reactions triggered by the
neutrons and measure their results to estimate the number and energy of the
neutrons. Since the imaging applications involve mostly slow neutrons, we will
focus our attention on their detection. 

Slow neutrons are detected by the nuclear reactions they trigger in ${}^{10}_5$
B, ${}^6_3$Li and ${}^3_2$He. The first two of these produce alpha particles
while the last one produces hydrogen atom. One can use non-relativistic energy
and momentum conservation laws to show that the energy of the alpha particle
in these reactions is \cite{turner2008atoms}
\begin{equation}\label{s5e1}
E = \frac{MQ}{m + M},
\end{equation}
where $m$ is the mass of the alpha particle, $M$ that of the daughter 
nucleus and $Q$ is the energy released in the reaction. The resulting alpha
particles are detected using proportional counters. Proportional counters
measure only the number of neutrons, not their energy spectrum. The reactions
used for detection are \cite{crane1991neutron}
\begin{equation}\label{s5e2}
{}^{10}_5\text{B}+{}^1_0\text{n} \rightarrow \begin{cases}
{}^7_3\text{Li}^\star + {}^4_2\text{He} + 2.31\text{ MeV} \\
{}^7_3\text{Li} + {}^4_2\text{He} + 2.79\text{ MeV}, 
\end{cases}
\end{equation}
where the $\star$ superscript in the first reaction indicates that the nucleus
is in an excited state. The other reactions are
\begin{eqnarray}
{}^6_3\text{Li}+{}^1_0\text{n}&\rightarrow&{}^3_1\text{H} + {}^4_2\text{He} +
4.78\text{ MeV} \label{s5e3} \\
{}^3_2\text{He}+{}^1_0\text{n}&\rightarrow&{}^3_1\text{H} + {}^1_1\text{H} +
0.765\text{ MeV} \label{s5e4} 
\end{eqnarray}
\section{Neutron dosimetry and shielding}\label{s6}
\section{Propagation of neutron beams}\label{s7}
\section{Neutron diffraction by crystals}\label{s8}
\section{2-d neutron imaging}\label{s9}
\section{Neutron tomography}\label{s10}
\section{Imaging of biologically important structures}\label{s11}
\section{Larmor precession based imaging}\label{s12}
\section{Polarization based imaging}\label{s13}
\section{Conclusion}\label{s14}
\bibliography{neutron}
\bibliographystyle{plain}
\end{document}
