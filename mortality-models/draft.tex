\documentclass{article}
\usepackage{amsmath, amssymb, amsthm, amsfonts}

\DeclareMathOperator{\pr}{P}

\title{A Review of Mortality Models}
\author{To be filled}
\date{Sep-2020}
\begin{document}
\maketitle
\section{Introduction}\label{s0}
Although it is very difficult to predict the behaviour of a member of a 
population one can often make statistical statements with much greater 
confidence. For example, it is impossible to predict when a particular 
radioactive atom will decay but one can make an accurate prediction of how 
many atoms will decay in a long enough time interval. Likewise, it is not
possible to predict the time of death of an individual although one reliably
predict how many people from a large enough group will survive past a certain
date in future. Actuaries knew this statistical fact long before the 
mathematical foundations of the theory of probability were laid. The 
mathematical models they to describe how deaths occur in a given society are
called `mortality models'.

The death of an individual in a population or the decay of a radioactive atom
is an event happening to a member of the population. Likewise, the failure of
a machine, an occurrence of an event, the duration of a certain phase in an
economy are all different manifestations of the same phenomenon in which 
something happens that significantly alters the future of the member of the
population. Death terminates an individual's existence, radioactive decay
changes the atom, failure renders a machine unusable and economies change 
after occurrences of certain cataclysmic events. Mortality of an individual
is akin to reliability of a machine. Life of an individual is similar to
the duration of a certain phase of business cycle. As a result, the 
theoretical tools used to study these phenomena are similar to each other. 
Mortality analysis in actuarial science is called survival analysis is 
statistics, reliability theory in engineering, duration analysis in economics 
and event history analysis in sociology.

In this article we shall review the development mathematical models to study
mortality in large enough populations of people.

\section{de Moivre's law}\label{s1}
An annuity is a lifelong payment made to an individual by a company. The 
company would want to estimate how much annuity its customers will consume
before they cease to exist. In order to estimate its yearly obligations, the
company would need to know what proportion of its customers will survive
past various points in the future. It to this problem that de Moivre 
\cite{de1731annuities} addresses his treatise. In order to estimate the value
of annuities one must consider the interest earned by the company's assets and
the probability of survival of its customers. In de Moivre's days the interest
rates were controlled by law. Therefore, he focused on estimating the 
probability of survival of the company's customers.

Developed in the 18th century, de Moivre's model was indeed quite simple. 
He assumed that there is a certain age, $\omega$ beyond which no human 
survives. Further, as time goes by the probability of a person's survival
decreases. The probability that a new born person will survive at least
$x$ years was proposed to be
\begin{equation}\label{s1e1}
s(x) = 1 - \frac{x}{\omega}, 
\end{equation}
where $0 \le x \le \omega$. The function $s$, called the survival function,
has the following characteristics
\begin{enumerate}
\item It is a decreasing function of $x$.
\item $s(x = 0) = 1$ and $s(x = \omega) = 0$. In between these age limits, it
is a linear function of $x$.
\end{enumerate}
de Moivre proposed the use of different $\omega$'s for different age groups, 
making his model piece-wise linear. We will examine a few immediate 
consequences of de Moivre's model and while doing so we will introduce the
notation common in actuarial literature \cite{jordan1967society}. The symbol
$(x)$ denotes life aged $x$. If $k$ is an estimate of the size of a population
born together then the number of people alive at age $x$ is $l_x = ks(x)$.
The number of people who will die in the interval $x$ to $x + 1$ is $d_x = l_x
- l_{x+1}$. The probability that $x$ will survive for another $n$ years is
\begin{equation}\label{s1e2}
{}_np_x = \frac{l_{x+n}}{l_x} = \frac{s(x+n)}{s(x)}.
\end{equation}
The probability that a person will die within next $n$ years of $x$ is
\begin{equation}\label{s1e3}
{}_nq_x = 1 - {}_np_x.
\end{equation}
It is common to denote ${}_1p_x$ as $p_x$ and ${}_nq_x$ as $q_x$. The `force
of mortality' is defined as
\begin{equation}\label{s1e4}
\mu(x) = -\frac{s^\prime(x)}{s(x)}, 
\end{equation}
where a $s^\prime$ denotes the derivative of $s$ with respect to $x$. The 
force of mortality is also called the hazard function in the statistical
literature. For de Moivre's law of mortality, it is easy to confirm that
\begin{eqnarray}
{}_np_x &=& \frac{\omega - (x + n)}{\omega - x} \label{s1e5} \\
{}_nq_x &=& \frac{n}{\omega - x} \label{s1e6} \\
\mu(x) &=& \frac{1}{\omega - x} \label{s1e7}
\end{eqnarray}
de Moivre himself did not consider his law to be an accurate description of
human mortality. He emphasized in his treatise that it was an approximation
useful to calculate the annuities. His piecewise linear approximation was a
valuable device to carry out practical computations in the eighteenth century.

\bibliography{draft}
\bibliographystyle{plain}
\end{document}

