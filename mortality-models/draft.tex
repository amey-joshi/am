\documentclass{article}
\usepackage{amsmath, amssymb, amsthm, amsfonts}
\numberwithin{equation}{section}
\DeclareMathOperator{\pr}{P}

\title{A Review of Mortality Models}
\author{To be filled}
\date{Sep-2020}
\begin{document}
\maketitle
\section{Introduction}\label{s0}
Although it is very difficult to predict the behaviour of a member of a 
population one can often make statistical statements with much greater 
confidence. For example, it is impossible to predict when a particular 
radioactive atom will decay but one can make an accurate prediction of how 
many atoms will decay in a long enough time interval. Likewise, it is not
possible to predict the time of death of an individual although one reliably
predict how many people from a large enough group will survive past a certain
date in future. Actuaries knew this statistical fact long before the 
mathematical foundations of the theory of probability were laid. The 
mathematical models they to describe how deaths occur in a given society are
called `mortality models'.

The death of an individual in a population or the decay of a radioactive atom
is an event happening to a member of the population. Likewise, the failure of
a machine, an occurrence of an event, the duration of a certain phase in an
economy are all different manifestations of the same phenomenon in which 
something happens that significantly alters the future of the member of the
population. Death terminates an individual's existence, radioactive decay
changes the atom, failure renders a machine unusable and economies change 
after occurrences of certain cataclysmic events. Mortality of an individual
is akin to reliability of a machine. Life of an individual is similar to
the duration of a certain phase of business cycle. As a result, the 
theoretical tools used to study these phenomena are similar to each other. 
Mortality analysis in actuarial science is called survival analysis is 
statistics, reliability theory in engineering, duration analysis in economics 
and event history analysis in sociology.

In this article we shall review the development mathematical models to study
mortality in large enough populations of people.

\section{de Moivre's law}\label{s1}
An annuity is a lifelong payment made to an individual by a company. The 
company would want to estimate how much annuity its customers will consume
before they cease to exist. In order to estimate its yearly obligations, the
company would need to know what proportion of its customers will survive
past various points in the future. It to this problem that de Moivre 
\cite{de1731annuities} addresses his treatise. In order to estimate the value
of annuities one must consider the interest earned by the company's assets and
the probability of survival of its customers. In de Moivre's days the interest
rates were controlled by law. Therefore, he focused on estimating the 
probability of survival of the company's customers.

Developed in the 18th century, de Moivre's model was indeed quite simple. 
He assumed that there is a certain age, $\omega$ beyond which no human 
survives. Further, as time goes by the probability of a person's survival
decreases. The probability that a new born person will survive at least
$x$ years was proposed to be
\begin{equation}\label{s1e1}
s(x) = 1 - \frac{x}{\omega}, 
\end{equation}
where $0 \le x \le \omega$. The function $s$, called the survival function,
has the following characteristics
\begin{enumerate}
\item It is a decreasing function of $x$.
\item $s(x = 0) = 1$ and $s(x = \omega) = 0$. In between these age limits, it
is a linear function of $x$.
\end{enumerate}
de Moivre proposed the use of different $\omega$'s for different age groups, 
making his model piece-wise linear. We will examine a few immediate 
consequences of de Moivre's model and while doing so we will introduce the
notation common in actuarial literature \cite{jordan1967society}. The symbol
$(x)$ denotes life aged $x$. If $k$ is an estimate of the size of a population
born together then the number of people alive at age $x$ is $l_x = ks(x)$.
The number of people who will die in the interval $x$ to $x + 1$ is $d_x = l_x
- l_{x+1}$. The probability that $x$ will survive for another $n$ years is
\begin{equation}\label{s1e2}
{}_np_x = \frac{l_{x+n}}{l_x} = \frac{s(x+n)}{s(x)}.
\end{equation}
The probability that a person will die within next $n$ years of $x$ is
\begin{equation}\label{s1e3}
{}_nq_x = 1 - {}_np_x.
\end{equation}
It is common to denote ${}_1p_x$ as $p_x$ and ${}_nq_x$ as $q_x$. The `force
of mortality' is defined as
\begin{equation}\label{s1e4}
\mu(x) = -\frac{s^\prime(x)}{s(x)}, 
\end{equation}
where a $s^\prime$ denotes the derivative of $s$ with respect to $x$. The 
force of mortality is also called the hazard function in the statistical
literature. The hazard function or the `force of mortality' is an individual's
susceptibility to die. Its reciprocal is the individual's resistance to death. 
de Moivre's law proposes that a person's resistance to die varies linearly over
time. That is,
\begin{equation}\label{s1e5}
\nu(x) = \frac{1}{\mu(x)} = \omega - x.
\end{equation}
If $x_0 = a, x_1 = a + d, x_2 = a + 2d, ..., x_n = a + nd$ then it is clear 
that $\nu(x_n)$ form an algebraic progression. This point may seem to be just a 
mathematical characteristic at this stage. We shall see in the next section that
it marked a departure from de Moivre's model to a more realistic one. For de 
Moivre's law of mortality, it is easy to confirm that
\begin{eqnarray}
{}_np_x &=& \frac{\omega - (x + n)}{\omega - x} \label{s1e6} \\
{}_nq_x &=& \frac{n}{\omega - x} \label{s1e7} \\
\mu(x) &=& \frac{1}{\omega - x} \label{s1e8}
\end{eqnarray}
de Moivre himself did not consider his law to be an accurate description of
human mortality. He emphasized in his treatise that it was an approximation
useful to calculate the annuities. His piecewise linear approximation was a
valuable device to carry out practical computations in the eighteenth century.

\section{Gompertz's law}\label{s2}
An immediate consequence of de Moivre's law is that an individual's resistance
to death diminishes algebraically as time goes by. The English actuary 
Benjamin Gompertz observed \cite{gompertz1825xxiv} that the resistance to 
death decreases geometrically instead. If one considers small enough time 
intervals then at the end of every interval a person's resistance to death 
reduces by the same proportion. Expressed mathematically, it means that
\begin{equation}\label{s2e1}
\frac{d\nu}{dx} = -k\nu(x),
\end{equation}
where the function $\nu$ was defined in equation \eqref{s1e5}. It immediately
follows that
\begin{equation}\label{s2e2}
\log\nu(x) = -kx + \log B,
\end{equation}
where $B$ is a constant of integration. The `force of mortality' or the
hazard function is therefore
\begin{equation}\label{s2e3}
\mu(x) = Be^{kx}.
\end{equation}
The constant $B$ is called baseline mortality and $k$ is called the 
senescent component.
An individual's susceptibility to die thus increases exponentially with age.
From equation \eqref{s1e4} and \eqref{s2e3}, the differential equation for
the survival function is
\begin{equation}\label{s2e4}
\frac{s^\prime(x)}{s(x)} = Be^{kx}.
\end{equation}
The solution of this ordinary differential equation is
\begin{equation}\label{s2e5}
s(x) = C\exp\left(\frac{B}{k}\exp(kx)\right),
\end{equation}
where $C$ is another constant of integration. The constants $B$ and $C$ in
equation \eqref{s2e5} are determined from the mortality data of the population
under study. Gompertz suggested that his formula is applicable for the age
group between $10-15$ years of age up to $55-60$. In order to extend it beyond
the age of $60$, Gompertz recommended that a different set of constants $B$
and $C$ be used.

Gompertz was aware that death in a population has at least two factors. One
of them is the deterioration in the individual's resistance to death. The 
other one is chance alone. The first factor depends on the individual's age
while the second factor is constant throughout. However, in the formulation
of his law of mortality, Gompertz considered only the first factor.

\section{Makeham's law and its generalisations}\label{s3}
William Makeham, the English mathematician and actuary, added a term to
Gompertz's law of resistance to death which the latter had acknowledged but
not included. Makeham's modification \cite{makeham1867law} of equation 
\eqref{s2e3} is
\begin{equation}\label{s3e1}
\mu(x) = A + Be^{kx},
\end{equation}
where $A$ is a constant independent of the person's age. Makeham's survival
function is
\begin{equation}\label{s3e2}
s(x) = C\exp\left(Ax + \frac{B}{k}\exp(kx)\right).
\end{equation}
Makeham's survival function has three constants $A, B$ and $C$ and therefore
can be fitted to emprically observed mortality data more effectively than
Gompertz's model. Makeham's law was often applied to mortality tables in the
age range beyond $20$ years to almost end of life\cite{jordan1967society}. 
In 1890\cite{makeham1890further}, Makeham proposed yet another modification to 
his law by adding a linear term to the hazard function or the force of 
mortality. Makeham thought that in addition to the two factors spotted by
Gompertz there is another one that grows linearly with age. It was proposed to 
be
\begin{equation}\label{s3e3}
\mu(x) = A_1 + A_2x + Be^{kx}.
\end{equation}
The resulting survival function is
\begin{equation}\label{s3e4}
s(x) = C\exp\left(A_1x + \frac{A_2}{2}x^2 + \frac{B}{k}\exp(kx)\right).
\end{equation}
Makeham's law, like its predecessor, Gompertz's law was used to construct
mortality tables of adults. Their simplicity allowed them to be used 
effectively up to as late as the 1940s. Gompertz law was used to construct
the 1937 Standard Annuity Table and Makeham's law for 1940 Annuity table
\cite{jordan1967society}. Both laws assumed that the force of mortality 
increases monotonically with time. However, it is commonly observed that 
infant mortality contradicts this assumption. In order to extend Makeham's
law of 1860 to infantile ages, Wilhem Lazarus \cite{lazarus1867ueber} added
an exponentially decreasing term. His law expressed the force of mortality
as 
\begin{equation}\label{s3e5}
\mu(x) = B_1 e^{-k_1x} + A + B_2e^{k_2 x}.
\end{equation}
The constants $B_1$ and $B_2$ are positive numbers. The corresponding survival 
function is
\begin{equation}\label{s3e6}
s(x) = 
C\exp\left(-\frac{B_1}{k_1}e^{-k_1x} + Ax + \frac{B_2}{k_2}e^{k_2x}\right)
\end{equation}

The way infants' susceptibility to die of diseases upsets Gompertz's and 
Makeham's assumption of monotonicity of the law of mortality, do does the 
spike in the death rate of young adults. Wars were frequent in Europe of the 
eighteenth century and the victims were disproportionately the young recruits.
Even otherwise, young adults are more prone to death by accidents. In order
to take into account this mortality hump in the young, Thiele 
\cite{thiele1871mathematical} introduced a gaussian term to Lazarus' model.
His law of mortality is
\begin{equation}\label{s3e7}
\mu(x) = B_1 e^{-k_1x} + Ae^{-a_1(x - a_2)^2} + B_2e^{k_2 x}.
\end{equation}
All the constants in this equation, namely $A, a_1, a_2, B_1, k_1, B_2$ and 
$k_2$ are positive numbers. The presence of the gaussian term prevents us
from writing a closed form expression of the survival function. However, the
ordinary differential equation for the survival function can be readily 
integrated numerically. Alternatively, if the initial conditions are available,
one can write its solution in terms of the error function.

We observe that all generalisations of Makeham's laws introduce additional
constants to the force of mortality and the hazard function. These constants
provide additional `degrees of freedom' to the functions used to fit the 
mortality data, resulting in a better fit.

Forfar, McCutcheon and Wilkie proposed a far wider generalisation of the
Gompertz and Makeham's laws. They defined a class of models, called GM models
after their inventors, with the mathematical form
\begin{equation}\label{s3e8}
\mu(x) = \sum_{i=0}^{r-1}\alpha_i x^i + 
\exp\left(\sum_{i=1}^{s-1}\beta_i x^i\right).
\end{equation}
They also used the convention that if $r = 0$ then the polynomial terms are 
absent and if $s = 0$ then the exponential terms do not appear. They called
the law of force of mortality of equation \eqref{s3e8} as $\mathrm{GM}(r,s)$.
It is easy to check that Gompertz's law is $\mathrm{GM}(0, 2)$, Makeham's
first law is $\mathrm{GM}(1,2)$ and Makeham's second law is $\mathrm{GM}(2,2)$.
The $\mathrm{GM}(0, 2)$, $\mathrm{GM}(2,2)$ and $\mathrm{GM}(1,3)$ are used
by the Continuous Mortality Investigation Bureau\cite{pitacco2016high} in the 
United Kingdom.

\section{Other mortality laws prior to the 20th century}\label{s4}
de Moivre's, Gompertz's and Makeham's laws of mortality were not the only
ones in vogue in the nineteenth century and prior to it. In this section we
mention a few notable ones without commenting on their merits. Lambert's
survival function \cite{lambert1776dottrina} was
\begin{equation}\label{s4e1}
s(x) = \left(\frac{a - x}{x}\right)^2 - b\left(e^{-x/c} - e^{-x/d}\right).
\end{equation}
It had four degrees of freedom. Babbage\cite{babbage1823tables} proposed a 
quadratic survival function
\begin{equation}\label{s4e2}
s(x) = c - bx - ax^2.
\end{equation}
Young proposed a polynomial of degree $40$ as the survival function in 
\cite{adler1866memoir} 1826. von Littrow\cite{von1832lebensversicherungen}
extended Babbage's model to polynomials of higher degree. Moser
\cite{moser1839gesetze} proposed a polynomial survival function with 
fractional powers of $x$. It has five degrees of freedom and had the form
\begin{equation}\label{s4e3}
s(x) = 1 - ax^{1/4} _ bx^{9/4} - cx^{17/4} - dx^{25/4} + ex^{33/4}.
\end{equation}
All these models were attempts to fit the mortality data with a continuous 
function. They were not rooted in demographic observations like the models
of Gompertz, Makeham, Lazarus or Thiele. As a result, they soon fell into
disuse after Gompertz proposed his model and others improved upon it.

\section{Heligman-Pollard model}\label{s5}
The term `odds' is a well-defined term in probability theory. When we are 
dealing with an experiment with two outcomes - that something happens or 
it does not - then it is common in certain fields to express the 
probabilities in terms of odds.  For example, if $p$ is the probability 
that an event will happen and $q = 1 - p$ is the probability that it will 
not happen then the odds of the experiment is $p/q$. 

Recall from equations \eqref{s1e2} and \eqref{s1e3} that $p_x$ is the 
probability that life aged $x$ will survive for another year and $q_x$ is
the probability of the complimentary event. Therefore, the odds of death in 
this case are $q_x/p_x$.  Heligman and Pollard \cite{heligman1980age} proposed 
a model that has the algebraic structure of Thiele's mortality law but 
applied to odds of survival. Mathematically it is expressed as
\begin{equation}\label{s5e1}
\frac{q_x}{p_x} = A^{(x+B)^C}+
D\exp\left(-E\log^2\left(\frac{x}{F}\right)\right) + GH^x.
\end{equation}
The model has eight parameters, $A, B, C, D, E, F, G$ and $H$. Of these, $A,
B, C, D \in (0, 1)$, $E > 0$, $F \in (15, \omega)$, $G \in (0, 1)$ and $H
> 0$. The constant $\omega$ is the hump maximum and it can be between $55$
and $100$. The Heligman-Pollard model was first developed for the Australian
population post the second world war. An advantage of Heligman-Pollard model 
is in biological interpretation of each of the three terms on the right hand 
side of equation \eqref{s5e1}.

The first term on the right hand side of equation \eqref{s5e1} represent 
infantile mortality. It is a fast declining exponential representing the 
rapid fall in mortality during infancy. $A$ is approximately equal to $q_1$ 
and is a measure of the mortality rate of a one year old child. $C$ is a 
measure of how quickly the new-born adjusts to its surroundings and gains 
immunity. A higher $C$ indicates the fast decline in infant mortality as the
child passes through infancy. When $B = 0$, $q_0 = 0.5$ irrespective of the
values taken by $A$ and $C$. The value of $q_0$ is closer to $q_1$ if $B$ is
higher for a given value of $C$. $B$ is thus an indicator of the central 
tendancy of infant mortality. 

The second term is gaussian in the logarithm of age. It represent the accident
mortality for young adults of both sexes and maternal mortality in young 
females. It covers the `accident hump` observed in all demographic data. It
is prominently observed between ages $10$ and $40$. The parameter $D$ controls 
the severity of accident hump, $E$ determines the spread of the hump and $F$ 
the location.  Using a statistical analogy, $E$ is a measure of spread and 
$F$ that of the central tendency. 

The third term traces its roots to the Gompertz model. It represents the
exponential increase in the deterioration of the body. The parameter $G$ 
represents the senescent mortality and $H$ the increase in the rate of that
process.

\bibliography{draft}
\bibliographystyle{plain}
\end{document}

