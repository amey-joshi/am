\documentclass{beamer}
\usefonttheme[onlymath]{serif}
\usepackage{graphicx}
\newcommand{\sor}{\mathbf{R}}
\usetheme{Boadilla}
\title{Optimisation Methods in Business Analytics}
\author{Amey Joshi}
\date{\today}
\begin{document}
\begin{frame}
\titlepage
\end{frame}
\begin{frame}
\frametitle{The mathematics of decision-making}
\begin{itemize}
\item To optimise is to find the `best' among a number of possibilities.
\item For example, when we select a school for our children, we consider 
several criteria:
\begin{itemize}
\item The quality of education,
\item The distance from home,
\item The peer group,
\item Extra-curricular activities, sports etc.,
\item The cost.
\end{itemize}
\item We choose the school that \textbf{best} meets \textbf{all} the criteria.
\item We follow similar process to find the right job, car, life-partner or
home. A large number of business decisions follow a similar pattern.
\item Optimisation is the mathematical model behind decision-making.
\end{itemize}
\end{frame}

\begin{frame}
\frametitle{A brief history of optimisation}
\begin{itemize}
\item Kepler conceived and used the `secretary problem' to select his second
wife.
\item Newton was interested in finding a shape that experiences least 
resistance while moving in air - a study that led to the development of the
Calculus of Variations.
\item The method of least squares by Gauss and Legendre in the 19th century.
\item Until the 20th century the problems were mostly of a scientific nature.
\end{itemize}
\end{frame}

\begin{frame}
\frametitle{Optimisation outside science}
\begin{itemize}
\item The socialist states required a model to describe their centralised 
economic planning. Leontif developed an input-output model, an early linear
program (LP).
\item Kantorovich put the input-output model in the form of an LP.
\item Stigler worked on the problem of designing the cheapest diet that met
the nutritional needs of US soldiers in the 1930s.
\item Dantzig developed Simplex algorithm to solve LP.
\begin{itemize}
\item Perhaps the world's most `commercially successful' algorithm.
\item Used in all spheres of life - shipping, scheduling, matching, industrial
processes, military operations, transport.
\item Mysterious behaviour - theory warns that it can be very slow, in
practice it is quite fast.
\end{itemize}
\end{itemize}
\end{frame}

\begin{frame}
\frametitle{The optimisation problem}
\begin{itemize}
\item Minimise $f: \sor^n \rightarrow \sor$ subject to
\begin{itemize}
\item $g_i(x) \ge 0$, $i = 1, \ldots, n$ and $g_i: \sor^m \rightarrow \sor$,
\item $h_j(x) = 0$, $j = 1, \ldots, m$ and $h_j: \sor^m \rightarrow \sor$.
\end{itemize}
\item The function $f$ to be minimised is called the \textbf{objective}
function.
\item The variables $x = (x_1, \ldots, x_n) \in \sor^n$ to be searched for
the minimum are called the \textbf{decision} variables.
\item The functions $g_i, h_j$ are called \textbf{constraints}.
\item The form is not restrictive.
\begin{itemize}
\item To maximise $f$, minimise $-f$.
\item A constraint of the form $g(x) <= a$ can be written as $-g(x) + a \ge 0$.
\end{itemize}
\item The general optimisation problem is very difficult to solve.
\end{itemize}
\end{frame}

\begin{frame}
\frametitle{Convex sets}
\begin{itemize}
\item If you select any two points in a set $S$, join them by a straight line and
if the line lies entirely in $S$ then the set is convex.
\begin{itemize}
\item A circle, a polygon, a sphere are all convex.
\item A circle with a hole, a torus, a heart-shaped figure are not convex.
\end{itemize}
\end{itemize}
\begin{figure}
\includegraphics[scale=0.3]{cardioid}
\caption{A non-convex set}
\end{figure}
\end{frame}

\begin{frame}
\frametitle{Convex functions}
\begin{itemize}
\item $f$ is a convex function if 
\begin{equation}
f\left(\frac{x_1 + x_2}{2}\right) \le \frac{f(x_1) + f(x_2)}{2},
\end{equation}
for all $x_1, x_2$. Function of average is $\le$ average of the function.
\end{itemize}
\begin{figure}
\includegraphics[scale=0.3]{f1}
\end{figure}
\end{frame}

\begin{frame}
\frametitle{Why the fuss about convexity?}
\begin{itemize}
\item If the objective function $f$ and the constraints $g_i, f_j$ in an
optimisation problem are convex then it is called a \textbf{convex optimisation}
problem.
\item Convexity is the boundary between tractable and intractable problems.
\item A minimum of a convex function is \textbf{the} global minimum.
\end{itemize}
\begin{figure}
\includegraphics[scale=0.3]{many}
\end{figure}
\end{frame}

\begin{frame}
\frametitle{Linear programming}
\begin{itemize}
\item If the objective function $f$ and the constraints $g_i, f_j$ in an
optimisation problem are linear then it is a \textbf{linear programming}
problem.
\item Linear? Fine. What is `programming'? A relic of the 1940s.
\item The set of all points $x$ which satisfy the constraints is called the
feasible set $F$.
\item The optimal point $x^\ast \in F$ is the one for which $f(x^\ast) \le
f(x)$ for all $x \in F$.
\item The feasible set is a polygon in 2d or a polyhedron in 3d or a convex
polytope in higher dimensions.
\item The point $x^\ast$ is \textbf{always} on the corner of the polytope.
\begin{itemize}
\item Simplex algorithm: hop from corner to corner in search of the minimum.
\item Interior point methods: cut through the interior of the polytope to
reach the minimum.
\end{itemize}
\end{itemize}
\end{frame}

\begin{frame}
\frametitle{The diet problem}
\begin{itemize}
\item A military problem from the 1930s. There are $77$ food items. For each
one of them we have data about the number of calories and amount of Ca, Fe,
vitamins A, B1, B2, B3 and C. 
\item We are also given how many calories and nutrients are needed by an
average man, every day.
\item \textbf{The problem}: Find the cheapest combinations of foods that meets
calorific and nutritional needs.
\item George Stigler and his team took 120 person days to solve it using
heuristic methods and desk calculators.
\item Simplex method implemented in Google OR tools on Google Colaboratory
took $0.5$ ms.
\end{itemize}
\end{frame}

\begin{frame}
\frametitle{Integer linear programming}
\begin{itemize}
\item What if you insist that some or all decision variables are integers?
\item The problem ceases to be convex. 
\begin{itemize}
\item If $m$ and $n$ are two integers in a set and if $m + n$ is odd then
clearly $(m + n)/2$ is not a member of the set.
\end{itemize}
\item Integer linear programming is $NP$-hard.
\item Solved using 
\begin{itemize}
\item Cutting planes method;
\item Branch and bound method;
\item Branch and cut.
\end{itemize}
\end{itemize}
\end{frame}

\begin{frame}
\frametitle{Examples of LP - Transportation Problem - 1}
\begin{itemize}
\item There are $m$ warehouses and $n$ outlets. Each warehouse has a certain 
amount of material available and each outlet has a certain demand for the 
material. Every pair of a warehouse and an outlet has a certain cost to 
transport the material.

\item Find the cheapest way to transport goods from the warehouses to the outlets
so that every outlet's demands are met and no warehouse supplies more than
it has.
\end{itemize}

\end{frame}

\begin{frame}
\frametitle{Examples of LP - Transportation Problem - 1}
\begin{itemize}
\item Decision variables: $x_{ij}$ the amount of material shipped from warehouse
$i$ to outlet $j$.
\item Objective function: 
\begin{equation}\nonumber
\sum_{i=1}^m\sum_{j=1}^n c_{ij}x_{ij}
\end{equation}
\item Constraints:
\begin{itemize}
\item If $a_1, \ldots, a_m$ are the amounts available in the warehouses then
\begin{equation}\nonumber
\sum_{j=1}^n x_{ij} \le a_i.
\end{equation}
\item If $d_1, \ldots, d_n$ are the demands of each outlet then
\begin{equation}\nonumber
\sum_{i=1}^m x_{ij} \le d_j.
\end{equation}
\item Each $x_{ij}$ should be $\ge 0$.
\end{itemize}
\end{itemize}
\end{frame}

\begin{frame}
\frametitle{Some more examples of LP}
\begin{itemize}
\item We are given a lorry with a certain capacity in terms of the weight it 
can carry. We are also given a set of items. Each item has a weight, a value 
and a quantity. The quantity is the number of items of that kind. Fill the 
lorry with a selection of items such that the value is maximised. (Knapsack
problem)

\item Consider a rail network connecting two cities by way of a number of 
intermediate cities, where each link of the network has a number assigned to it 
representing its capacity. Assuming a steady state condition, find a maximal 
flow from one given city to the other. (Max-flow problem)

\item There are $m$ agents and $n$ tasks. An agent can do one or more tasks.
Find an assignment of the agents such that maximum number of tasks are completed
and no task is done be more than one agent. (Assignment problem)
 
\end{itemize}
\end{frame}

\begin{frame}
\frametitle{Beyond LP}
\begin{itemize}
\item When the objective function is a quadratic form we have the quadratic 
programming problem. Used in portfolio optimisation.
\item Even OLS regression is an optimisation problem.
\item Logistic regression uses MLE solved by an optimiser.
\item Gradient descent, Adaptive gradient descent, conjugate gradient descent,
xgboost are all optimisation algorithms used in ML algorithms. All of them 
minimise an error term.
\item Several optimisation algorithms are inspired by nature. For example,
simulated annealing, genetic algorithms, swarm-based algorithms etc.
\end{itemize}
\end{frame}

\begin{frame}
\frametitle{To sum up ...}
\begin{itemize}
\item Optimisation mimics human thinking; no surprise that so many AI
algorithms use optimisation underneath;
\item The boundary between easy and hard problem is determined by convexity;
\item Linear programming is far more pervasive than we sometimes imagine.
\end{itemize}
\end{frame}
\end{document}
