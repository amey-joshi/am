\documentclass[11pt]{article}
\usepackage{amsmath, amsfonts, amssymb, amsthm}
\usepackage{physics, mathtools, graphicx}
\numberwithin{equation}{section}
\linespread{1.2}
\title{Young's Experiment}
\author{To be added}
\begin{document}
\maketitle
\section{Introduction}
Sir Isaac Newton proposed that light is a stream of tiny particles emitted
from a luminous body. His influence on the world of physicists ensured that
his corpuscular theory of light was favored over the rival wave theory for 
almost two centuries. Thomas Young's double-slit experiment that the most
convincing proof that light behaves like a wave. The experiment consists in
passing light through two slits cut out in a screen and observing the 
illumination on another distant screen. When the slits are wide they create
two separate illuminated regions on the screen. When they are narrow, the
two bright regions fade away and a pattern of alternating dark and bright
fringes appears. Young reasoned that these fringes are a result of constructive
and destructive interference of light waves. Light emerging from narrow slits
diffracts into the geometrical shadow. The interference fringes are produced by
the diffracted waves. Both diffraction and interference are phenomena associated
with waves.

Young's experiment showed that when the slits are wide enough, light indeed
travels in a straight line like Newtonian particles described in Newton's first
law. This is the realm of `geometrical optics'. In this realm, light casts
sharp shadows. However, when the slits are narrow, the shadows are no longer
sharp. A penumbra appears and as the slits narrow down further the penumbra 
evolves into an interference pattern as shown in figure \ref{f1}.

\begin{figure}
\centering
\includegraphics[scale=0.5]{young}
\caption{Young's experiment. Figure from reference \cite{wiki1}}\label{f1}
\end{figure}

Young explained his experiment by proposing that the two slits act as sources
of `secondary waves' which interfere on the screen. The wavelength $\lambda$
of the two sources is identical. When the path difference of the two waves
at a point on the screen is an integral multiple of $\lambda$, the waves
interfere constructively and result in a bright band. On the other hand, when
the path difference is an odd multiple of $\lambda/2$, the waves interfere
destructively, resulting in a dark band.

Most common sources of light produce a range of wavelengths. As a result, the
interference bands produced on the screen are not as sharp as one would expect
were the light monochromatic. They also produce light waves whose phase 
changes randomly at an extremely fast rate. Therefore, if we use two separate
light sources illuminating the two slits, there is no fixed path difference
at any point on the screen and therefore no interference pattern is observed.
In order to overcome this problem, the two slits are always illuminated by 
the same source of light. This ensures that irrespective of the random phase
fluctuations at the source, the path difference is a function of the 
experiment's geometry alone.

If $d$ is the distance between the slits and $D$ is the separation between
the screen that has the slits and the screen on which we observe the 
interference pattern then the distance between two consecutive bright (or
dark) fringes is given by \cite{jenkins1937fundamentals}
\begin{equation}\label{s1e1}
\delta y = \lambda \frac{D}{d},
\end{equation}
where $\lambda$ is the wavelength of light.

Although Young's experiment demonstrated the wave nature of light in 1801, it 
was not clear in at that time what exactly was oscillating. The answer to that
question had to wait six decades until James Clerk Maxwell proposed his 
equations governing the electric and magnetic phenomena. His genius lay in 
adding a `displacement current' term to Ampere's law and realizing that the 
other known laws like Gauss law, Faraday's law, the fact that there are
no magnetic monopoles together can explain all electric and magnetic 
phenomena. An immediate consequence of his equations was his wave equation
which showed the existence of electromagnetic waves. Maxwell proposed that
light is a form of an electromagnetic wave and that gave a theoretical
explanation to Young's experiment.

\section{Maxwell equations}
Maxwell equations, in SI units, are
\begin{eqnarray}
\div\vb{D} &=& \rho_f \label{s2e1} \\
\curl\vb{E} &=& -\pdv{\vb{B}}{t} \label{s2e2} \\
\div\vb{B} &=& 0 \label{s2e3} \\
\curl\vb{H} &=& \vb{J}_f + \pdv{\vb{D}}{t}, \label{s2e4}
\end{eqnarray}
where $\vb{E}$ is the electric field, $\vb{B}$ is the magnetic field,
$\vb{D}$ is the electric displacement and $\vb{H}$ is the auxilary magnetic
vector. $\rho_f$ is the density of free charges and $\vb{J}_f$ is the 
density of free current. Equation \eqref{s2e1} is Gauss' law. Equation
\eqref{s2e2} is Faraday's law. Equation \eqref{s2e3} is the mathematical
expression of the fact that there are no magnetic monopoles. Equation
\eqref{s2e4} is Ampere's law. The term $\partial\vb{D}/\partial t$ in equation
\eqref{s2e4} is called the displacement current. Adding it to Ampere's law
was Maxwell's correction to the law. We will show that this term is critical
in getting the equation of electromagnetic waves.

The electric displacement and the electric field
are related by the relation
\begin{equation}\label{s2e5}
\vb{D} = \epsilon_0\vb{E} + \vb{P},
\end{equation}
where $\epsilon_0$ is the permittivity of free space and $\vb{P}$ is the
polarization of the medium. It is the number of electric dipoles per unit
volume of the material. The volume of the material is chosen to be small at
a macroscopic scale but large enough at a microscopic scale so that it contains
a very large number of molecules of the material. The magnetic field $\vb{B}$
is related to the field $\vb{H}$ by the relation,
\begin{equation}\label{s2e6}
\vb{H} = \mu_0^{-1}(\vb{B} - \vb{M}),
\end{equation}
where $\mu_0$ is the permeability of free space and $\vb{M}$ is the 
magnetization of the medium. It is the number of magnetic dipoles per unit
volume of the material. The volume of the material is chosen in a manner 
similar to the one used to define $\vb{P}$. 

There are no electric or magnetic dipoles in vacuum so that equations
\eqref{s2e5} and \eqref{s2e6} become
\begin{eqnarray}
\vb{D} &=& \epsilon_0\vb{E} \label{s2e7} \\
\vb{H} &=& \mu_0^{-1}\vb{B} \label{s2e8}
\end{eqnarray}
It is also common to write equations \eqref{s2e5} and \eqref{s2e6} as
\begin{eqnarray}
\vb{D} &=& \epsilon\vb{E} \label{s2e9} \\
\vb{H} &=& \mu^{-1}\vb{B}, \label{s2e10}
\end{eqnarray}
where $\epsilon$ and $\mu$ are permittivity and permeability of media. These
equations are applicable to linear, isotropic media. A general form of these
equations is
\begin{eqnarray}
D_i &=& \epsilon_{ij}E_j \label{s2e11} \\
B_i &=& \mu_{ij}H_j \label{s2e12}
\end{eqnarray}
where $\epsilon_{ij}$ is the permittivity tensor and $\mu_{ij}$ is the
permeability tensor of the medium. 

In this article, we will consider electromagnetic waves in vacuum where there 
are no free charges or currents. Therefore $\rho_f = 0$ and $\vb{J}_f= 0$.
Maxwell equations, in this case, simplify to
\begin{eqnarray}
\div\vb{E} &=& 0 \label{s2e13} \\
\curl\vb{E} &=& -\pdv{\vb{B}}{t} \label{s2e14} \\
\div\vb{B} &=& 0 \label{s2e15} \\
\curl\vb{B} &=& \mu_0\epsilon_0\pdv{\vb{E}}{t} \label{s2e16}
\end{eqnarray}
Taking the curl of equation \eqref{s2e14} and using equation \eqref{s2e16}
in the result, we get
\begin{equation}\label{s2e17}
\laplacian\vb{E} = \mu_0\epsilon_0\pdv[2]{\vb{E}}{t}.
\end{equation}
This is the equation of an electromagnetic wave in vacuum. Its speed is
\begin{equation}\label{s2e18}
c_0 = \frac{1}{\sqrt{\epsilon_0\mu_0}}.
\end{equation}
If we take the curl of equation \eqref{s2e16} and use equation \eqref{s2e14},
we get
\begin{equation}\label{s2e19}
\laplacian\vb{B} = \mu_0\epsilon_0\pdv[2]{\vb{B}}{t}.
\end{equation}
Thus, the magnetic field $\vb{B}$ satisfies an identical equation. The key
physical idea governing the wave equations \eqref{s2e17} and \eqref{s2e19}
is that a changing magnetic field produces an electric field (Faraday's law
of equation \eqref{s2e14}) and a changing electric field produces a 
magnetic field (Ampere's law with Maxwell's correction of equation 
\eqref{s2e16}). These two laws together ensure that the electric and magnetic
waves sustain each other and produce an electromagnetic wave.

\section{Electromagnetic waves}\label{s3}
\section{Finite Difference Time Domain Method}\label{s4}

\bibliographystyle{plain}
\bibliography{em}
\end{document}

