\chapter{Functions on Euclidean Space}\label{c1}
\section{Norm and Inner Product}\label{c1s1}
$\sor^n$ has points $x = (x^1, \ldots, x^n)$ such that if $\alpha \in \sor$
then $\alpha x = (\alpha x^1, \ldots, \alpha x^n)$ is also a point in $\sor^n$
and so is $x + y = (x^1 + y^1, \ldots x^n + y^n)$. $\sor^n$ is a linear vector
space. 
\begin{defn}\label{c1s1d1}
The norm of a vector $x \in \sor^n$ is defined as
\[
\norm{x} = \left((x^1)^2 + \cdots + (x^n)^2\right)^{1/2}.
\]
\end{defn}

\begin{defn}\label{c1s1d2}
The inner product of vectors $x, y \in \sor$ is defined as
\[
\ip{x}{y} = \abs{\sum_{i=1}^n x^i y^i }.
\]
\end{defn}

\begin{lem}\label{c1s1l1}
$\norm{x}^2 = \ip{x}{x}$.
\end{lem}

\begin{lem}\label{c1s1l2}
$\ip{x}{y} = \ip{y}{x}$.
\end{lem}

\begin{lem}\label{c1s1l3}
$\ip{\alpha x}{y} = \alpha\ip{y}{x}$ and $\ip{x}{\alpha y} = \alpha\ip{x}{y}$
for all $\alpha \in \sor$.
\end{lem}

\begin{lem}\label{c1s1l4}
$\ip{x}{y_1 + y_2} = \ip{x}{y_1} + \ip{x}{y_2}$ and $\ip{x_1 + x_2}{y} = 
\ip{x_1}{y} + \ip{x_2}{y}$.
\end{lem}

\begin{lem}[Polarisation identity]\label{c1s1l5}
\[
\ip{x}{y} = \frac{\norm{x + y}^2 - \norm{x - y}^2}{4}.
\]
\end{lem}
\begin{proof}
Follows immediately from lemmas \ref{c1s1l1} and \ref{c1s1l2}.
\end{proof}

The norm has the following properties.
\begin{thm}\label{c1s1t1}
If $x, y \in \sor^n$ and $\alpha \in \sor$ then
\begin{enumerate}
\item $\norm{x} \ge 0$ and $\norm{x} = 0$ iff $x = 0$.
\item Cauchy-Schwarz inequality
\[
\ip{x}{y} \le \norm{x}\norm{y}.
\]
the equality is true if $x = \lambda y$ for some $\lambda \in \sor$.
\item $\norm{x + y} \le \norm{x} + \norm{y}$.
\item $\norm{\alpha x} = \abs{\alpha}\norm{x}$ for all $\alpha \in \sor$.
\end{enumerate}
\end{thm}
\begin{proof}
$\norm{x}$ is the positive square root of the sum of non-negative reals. 
Therefore it is non-negative. If $x = 0$ then clearly $\norm{x} = 0$. If
$\norm{x} = 0$ then the sum of $n$ non-negative numbers is zero, which is
possible only if all of them are zero.

Consider $\norm{x - \lambda y}^2 = \ip{x - \lambda y}{x - \lambda y}$
so that
\[
\norm{x - \lambda y}^2 = \ip{x}{x} - 2\lambda\ip{x}{y}  + \lambda^2\ip{y}{y}
\]
We showed previously that $\norm{x - \lambda y} \ge 0$ and that it is zero
iff $x - \lambda y = 0$. If $\norm{x - \lambda y}^2 > 0$ then then the quadratic
on the rhs has no real roots, in which case the discriminant
\[
4\lambda^2\ip{x}{y}^2 - 4\lambda^2\norm{x}^2\norm{y}^2 < 0
\]
from which it follows that
\[
\ip{x}{y} < {\norm{x}}{\norm{y}}.
\]
If, on the other hand, $\norm{x - \lambda y} = 0$ then the quadratic has real
roots in which case $\ip{x}{y} \ge {\norm{x}}{\norm{y}}$. However, if $\norm{x -
\lambda y} = 0$ then $x = \lambda y$ and then $\ip{x}{y} = {\norm{x}}
{\norm{y}}$. In either case, $\ip{x}{y} \le {\norm{x}} {\norm{y}}$. Now suppose
that $\ip{x}{y} = \norm{x}\norm{y}$ then the discriminant of the quadratic
is zero which indicates that real roots exist and that $x = \lambda y$.

$\norm{x + y}^2 = \ip{x+y}{x+y} = \norm{x}^2 + 2\ip{x}{y} + \norm{y}^2$. Using
the Cauchy-Schwarz inequality, $\norm{x + y}^2 \le \norm{x}^2 + 2\norm{x}
\norm{y} + \norm{y}^2 = (\norm{x} + \norm{y})^2$.

The fourth statement follows immediately from the definitions of the scalar
product and the norm.
\end{proof}

