\chapter{Spaces}\label{c1}
\section{Fields}\label{c1s1}
\begin{defn}\label{c1s1d1}
A set $F$ with binary operations $+$ and $\cdot$ defined on it is a
field is
\begin{enumerate}
\item (A) Properties of the operation `$+$':
\begin{enumerate}
\item For every $\alpha, \beta \in F$, $\alpha + \beta \in F$,
\item For every $\alpha, \beta \in F$, $\alpha + \beta = \beta +
\alpha$,
\item For all $\alpha, \beta, \gamma \in F$, $\alpha + (\beta + 
\gamma) = (\alpha + \beta) + \gamma$,
\item There exists a member $0$ such that $\alpha + 0 = \alpha$ for
all $\alpha \in F$,
\item For every $\alpha \in F$, there exists a member $-\alpha$
such that $\alpha + (-\alpha) = 0$.
\end{enumerate}
\item (B) Properties of the operation `$\cdot$':
\begin{enumerate}
\item For every $\alpha, \beta \in F$, $\alpha \cdot \beta \in F$,
\item For every $\alpha, \beta \in F$, $\alpha \cdot \beta = \beta 
\cdot \alpha$,
\item For all $\alpha, \beta, \gamma \in F$, $\alpha \cdot (\beta 
\cdot \gamma) = (\alpha \cdot \beta) \cdot \gamma$,
\item There exists a member $0$ such that $\alpha \cdot 1 = \alpha$ for
all $\alpha \in F$,
\item For every $\alpha \ne 0 \in F$, there exists a member 
$\alpha^{-1}$ such that $\alpha \cdot \alpha^{-1} = 1$.
\end{enumerate}
\item The operation $\cdot$ is distributive over $+$. That is $\alpha
\cdot (\beta + \gamma) = \alpha \cdot \beta + \alpha \cdot \gamma$
for all $\alpha, \beta, \gamma \in F$.
\end{enumerate}
\end{defn}

\begin{rem}
Members of a field are called scalars.
\end{rem}

\subsection{Exercises}
\begin{enumerate}
\item We will prove the following assertions.
\begin{enumerate}
\item $0 + \alpha$ by commutativity of `+' is the same as $\alpha + 0
= \alpha$.
\item $\alpha + \beta = \alpha + \gamma \Rightarrow -\alpha + (\alpha
+ \beta) = -\alpha + (\alpha + \gamma) \Rightarrow (-\alpha + \alpha)
+ \beta = (-\alpha + \alpha) + \gamma$. Now, by commutativity of `+',
$\alpha + (-\alpha) = 0 \Rightarrow -\alpha + \alpha = 0$. Therefore,
$0 + \beta = 0 + \gamma \Rightarrow \beta = \gamma$.
\item $\alpha + (\beta - \alpha) = \alpha + (-\alpha + \beta) = 
\alpha + (-\alpha) + \beta = 0 + \beta = \beta$.
\item Since $0$ is an additive identity, $0 + 0 = 0 \Rightarrow
\alpha\cdot(0 + 0) = \alpha\cdot 0 \Rightarrow \alpha\cdot 0 + 
\alpha\cdot 0 = \alpha\cdot 0$. Since $\alpha\cdot 0$ is a member of
$F$, it has an additive inverse $-\alpha\cdot 0$. Thus,
$-\alpha\cdot 0 + (\alpha\cdot 0 + \alpha\cdot 0) = 
-\alpha\cdot 0 + \alpha\cdot 0 \Rightarrow
(-\alpha\cdot 0 + \alpha\cdot 0) + \alpha\cdot 0 = 0 \Rightarrow
0 + \alpha\cdot 0 = 0 \Rightarrow \alpha\cdot 0 = 0.$
The claim $0\cdot\alpha = 0$ follows from commutativity of $\cdot$.
\item We observe that $(-1)\alpha + \alpha = \alpha(-1) + \alpha(1)
= \alpha(-1 + 1) = \alpha 0 = 0$. Thus, $(-1)\alpha$ is the additive
inverse of $\alpha$. $(-1)\alpha = -\alpha$ follows from the
uniqueness of the additive inverse.
\item $(-\alpha)(-\beta) + (-\alpha)(\beta) = -\alpha(-\beta + \beta)
= -\alpha \cdot 0 = 0$. Thus, $(-\alpha)(-\beta)$ is the additive 
inverse of $(-\alpha)(\beta)$. We also observe that $\alpha\beta + 
(-\alpha)\beta = (\alpha - \alpha)\beta = 0\cdot\beta = 0$. Thus,
$\alpha\beta$ is also an additive inverse of $(-\alpha)(\beta)$. The
claim follows from the uniqueness of additive inverse. 

This claim used the fact that $(\alpha + \beta)\gamma = \alpha\gamma
+ \beta\gamma$. It follows from commutativity of the two operations.

\item Let $\alpha\beta = 0$. If $\alpha \ne 0$ then $\alpha^{-1}$
exists. Multiply both sides of the equation by $\alpha^{-1}$ to get
$\beta = 0$. Similarly, $\beta \ne 0$ gives $\alpha = 0$. The equality
is also true when both $\alpha$ and $\beta$ are zero.
\end{enumerate}
\end{enumerate}
