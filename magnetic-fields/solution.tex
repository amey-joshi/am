\documentclass{article}
\usepackage{amsmath, amsthm, amssymb, amsfonts}
\title{Two problems in Electrodynamics}
\begin{document}
\begin{enumerate}
\item Consider a finite solenoid of length $L$ and cross-sectional area 
$A = \pi R^3$. Let there be $n$ turns of the wire per unit length and 
let $I$ be the current passing through them.

We know that $\nabla\cdot\vec{B}$ is always zero. Imagine a hemisphere whose
base is perpendicular to the solenoid's axis and intersects it. Let the
hemisphere be of radius $a$. Let $H$ denote the surface of the hemisphere
and $V$ denote its volume. Then,
\[
\int_V \nabla\cdot\vec{B}dV = 0 \Rightarrow \int_H \vec{B}\cdot\hat{n}da = 0,
\]
where $\hat{n}$ is the unit outward normal. We can write the surface integral 
as the sum over the base $H_1$ and the curved surface $H_2$. Thus,
\[
\int_{H_1}\vec{B}\cdot\hat{n}da = -\int_{H_2}\vec{B}\cdot\hat{n}da.
\]
As $a \rightarrow \infty$, the field drops as $1/a^3$ while the surface area
rises as $a^2$. Effectively, the integrand drops as $1/a$. Thus, as $a 
\rightarrow \infty$, the integrand on the right hand side of the previous
equation vanishes and so does the integral. Therefore, as $a \rightarrow 
\infty$,
\[
\int_{H_1}\vec{B}\cdot\hat{n}da = 0.
\]
The integrand in this integral is also called the magnetic flux. As $\vec{B}
= 0$ we know that the field lines inside the solenoid are in a direction
opposite to that outside. If $\Phi_i$ is the flux due to the field inside the
solenoid and $\Phi_o$ is that due to the field outside then the previous
equation can be written as 
\[
\Phi_i + \Phi_o = 0.
\]
Further, we know that $\Phi_i = \mu_0 nIA$. Therefore,
\[
\Phi_o = -\mu_0 nIA.
\]
We now consider a square on the surface $H_1$ whose centre lies on the 
axis of the solenoid. If $B$ is the field in this square then it contributes
a flus $BL^2$. The portion of $H_1$ beyond this square does not contribute
significantly because the field drops off as $r^{-3}$. Thus, close enough
to the solenoid the magnitude of the field is
\[
B_{\text{close}} \approx \frac{\mu_0 nIA}{L^2},
\]
and far away from the solenoid it is essentially zero.

\item Given that a charge $Q$ is uniformly distributed on a sphere of radius
$R$. Therefore, its surface charge density is
\begin{equation}\label{p2e1}
\sigma = \frac{Q}{4\pi R^2}.
\end{equation}
The sphere is rotating at a uniform angular velocity of $\vec{\omega}$. We want
to find the magnetic field at an arbitrary point, say $P$. We choose to set
the origin of our coordinate axes to coincide with the centre of the sphere and
we align the $z$ axis to lie along the line $OP$. Let the angle between the
constant vector $\vec{\omega}$ and the $z$ axis by $\psi$. Note that $\psi$
is a constant in our problem. We also align our $x$-axis so that $\vec{\omega}$
lies in the $xz$ plane. We can then write it as
\begin{equation}\label{p2e2}
\vec{\omega} = \omega\sin\psi\hat{i} + \omega\cos\psi\hat{k}.
\end{equation}

The rotation of a charged sphere gives rise to a current density $\vec{J} = 
\sigma\vec{v}$, where $\sigma$, the charge density is given by equation 
\eqref{p2e1} and $\vec{v}$ is the velocity of the point on the sphere at 
which we are measuring $\vec{J}$. Since $\vec{v} = \vec{\omega}\times\vec{r}$
for a uniformly rotating sphere,
\[
\vec{J} = \sigma\vec{\omega} \times \vec{r}.
\]
From equation \eqref{p2e2} and the fact that $\vec{r} = R\sin\theta\cos\phi
\hat{i} + R\sin\theta\sin\phi\hat{j} + R\cos\theta\hat{k}$, we get
\begin{eqnarray}
\vec{J}&=&\sigma R\omega\Big(-\cos\psi\sin\theta\sin\phi\hat{i} \nonumber \\
 & & (\cos\psi\sin\theta \cos\phi - \sin\psi\cos\theta)\hat{j} + \nonumber  \\
 & & \sin\psi\sin\theta\sin\phi\hat{k}\Big)\label{p2e3}
\end{eqnarray}
In order to find the magnetic field due to this current distribution, we 
first find the vector potential. If $\vec{s}$ is the position vector of
the point $P$ then 
\begin{equation}\label{p2e4}
\vec{A}(\vec{s}) = \frac{\mu_0}{4\pi}\int\frac{\vec{J}}{|\vec{s} - \vec{x}|}
da,
\end{equation}
where $\vec{x}$ is a point on the sphere and $da = R^2\sin\theta d\theta d\phi$
is the `area element' on the sphere's surface. From equations \eqref{p2e3}
and \eqref{p2e4} it is clear that we have three integrals to evaluate, one for 
each component. Thus,
\begin{eqnarray}
A_x &=& -\frac{\mu_0}{4\pi}\int_{S}\frac{\sigma R\omega\cos\psi\sin\theta\sin\phi}
{|\vec{s} - \vec{x}|} R^2\sin\theta d\theta d\phi \label{p2e5} \\
A_y &=& \frac{\mu_0}{4\pi}\int_{S}\frac{\sigma R\omega(\cos\psi\sin\theta \cos\phi - \sin\psi\cos\theta)}
{|\vec{s} - \vec{x}|} R^2\sin\theta d\theta d\phi \label{p2e6} \\
A_z &=& \frac{\mu_0}{4\pi}\int_{S}\frac{\sigma R\omega\sin\psi\sin\theta\sin\phi\sin\theta\sin\phi}
{|\vec{s} - \vec{x}|} R^2\sin\theta d\theta d\phi \label{p2e7} 
\end{eqnarray}
The subscript $S$ indicates that the integration is over the surface of the
sphere. Now, the $\theta$ integral goes from $0$ to $\pi$ and the $\phi$ 
integral goes from $0$ to $2\pi$. Since
\begin{eqnarray*}
\int_0^{2\pi}\sin\phi d\phi &=& 0 \\
\int_0^{2\pi}\cos\phi d\phi &=& 0
\end{eqnarray*}
the only surviving integral among equations \eqref{p2e5} to \eqref{p2e6} is
\[
A_y = -\frac{\mu_0}{4\pi}\int_0^\pi 2\pi 
\frac{\sigma R\omega(\sin\psi\cos\theta)}{|\vec{s} - \vec{x}|} R^2\sin\theta d\theta 
\]
The vector potential is
\begin{equation}\label{p2e8}
\vec{A}(\vec{s}) = -\frac{\mu_0 R^3\sigma\omega\sin\psi}{2}
\int_0^\pi \frac{\cos\theta\sin\theta d\theta}{|\vec{s} - \vec{x}|}\hat{j}.
\end{equation}
From the geometry of the problem it is clear that
\[
|\vec{s} - \vec{x}| = \sqrt{s^2 + R^2 - 2Rs\cos\theta}
\]
so that
\begin{equation}\label{p2e9}
\vec{A}(\vec{s}) = -\frac{\mu_0 R^3\sigma\omega\sin\psi}{2}
\int_0^\pi \frac{\cos\theta\sin\theta d\theta}
{\sqrt{s^2 + R^2 - 2Rs\cos\theta}}\hat{j}.
\end{equation}

In order to carry out this integration, put $u = \cos\theta$ so that
\begin{equation}\label{p2e10}
\vec{A}(\vec{s}) = -\frac{\mu_0 R^3\sigma\omega\sin\psi}{2}
\int_{-1}^1\frac{udu}{\sqrt{R^2 + s^2 - 2Rsu}}\hat{j}
\end{equation}
The evaluation of this integral needs care. We first evaluate the corresponding
indefinite integral,
\[
I = \int \frac{udu}{\sqrt{R^2 + s^2 - 2Rsu}}
\]
and introduce the variable $v^2 = R^2 + s^2 - 2Rsu$ so that
\[
du = -\frac{vdv}{Rs}
\]
and hence
\begin{eqnarray*}
I &=& -\int \frac{v}{Rs}\frac{R^2 + s^2 - v^2}{2Rs}\frac{dv}{v} \\
 &=& -\int (R^2 + s^2 - v^2)dv \\
 &=& -\frac{1}{2R^2s^2}\left((R^2+s^2)v - \frac{v^3}{3}\right) \\
 &=& -\frac{v}{2R^2s^2}\left(R^2+s^2 - \frac{R^2+s^2-2Rsu}{3}\right) \\
 &=& -\frac{v}{3R^2s^2}(R^2 + s^2 + Rsu) \\
 &=& -\frac{R^2+s^2+Rsu}{3R^2s^2}\sqrt{R^2 + s^2 - 2Rsu}
\end{eqnarray*}
Substituting this in equation \eqref{p2e10} we get
\begin{equation}\label{p2e11}
\vec{A} = \frac{\mu_0R^3\sigma\omega\sin\psi}{2}
\left[\frac{R^2+s^2+Rsu}{3R^2s^2}\sqrt{R^2 + s^2 - 2Rsu}\right]_{-1}^1\hat{j}.
\end{equation}
The square root takes a value that depends on whether $s > R$ or $s < R$. We
also observe that $-\omega s \sin\psi\hat{j} = \vec{\omega}\times\vec{s}$ so 
that
\begin{equation}\label{p2e12}
\vec{A}(\vec{s}) = \begin{cases}
\frac{\mu_0 R\sigma}{3}(\vec{\omega} \times \vec{s}) \text{ if } s < R \\
\frac{\mu_0 R^4\sigma}{3s^2}(\vec{\omega} \times \vec{s}) \text{ if } s > R
\end{cases}
\end{equation}

We now rotate the axes so that $\vec{\omega}$ is along the $z$-axis whence
\begin{eqnarray}
\vec{A}(r, \theta, \phi)&=& \frac{\mu_0R\omega\sigma}{3}r\sin\theta\hat{\phi} \text{ if } r \le R \nonumber \\
 &=& \frac{\mu_0R^4\omega\sigma}{3}\frac{\sin\theta}{r^2}\hat{\phi} \text{ if } r \ge R \label{p2e13}
\end{eqnarray}
Therefore, the magnetic field is
\begin{eqnarray}
\vec{B}(\vec{r}) &=& \frac{2}{3}\mu_0 R\sigma\vec{\omega} \text{ if } r \le R \nonumber \\
 &=& \frac{\mu_0R^4\omega\sigma}{3}\left(\frac{2\cos\theta}{r^3}\hat{r} + \frac{\sin\theta}{r^3}\hat{\theta}\right) \text{ if } r \ge R \label{p2e14}
\end{eqnarray}

We are now in a position to answer the four questions related to this
problem.
\begin{enumerate}
\item[(a)] Since $\vec{J} = \sigma\vec{v} = \sigma\vec{\omega}\times\vec{r}
= \vec{M}\times\vec{r}$, where the constant vector $\vec{M} = 
\sigma\vec{\omega}$. In order to get the desired form of $\vec{J} = 
\vec{M} \times \Theta$, we need to find a function $\Theta$ such 
that $\nabla\Theta = \vec{r}$. One such choice is $\Theta(x, y, z) = 
x + y + z$ or $\Theta(x - R, y - R, z - R) = x - R + y - R + z - R$, 
both of which that the same gradient.
\item[(b)] The magnetic moment is also defined through the relation,
\[
\vec{A} = \frac{\mu_0}{4\pi} \frac{\vec{m}\times\vec{r}}{r^3}.
\]
Comparing this with the second of equations \eqref{p2e12} we get
\[
\vec{m} = \frac{4\pi}{3}R^4\sigma\vec{\omega}.
\]
\item[(c)] The vector potential is given by equation \eqref{p2e13}.
\item[(d)] The magnetic field is given by equation \eqref{p2e14}.
\end{enumerate}

\end{enumerate}
\end{document}
